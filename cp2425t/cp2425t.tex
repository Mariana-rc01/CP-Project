\documentclass[11pt, a4paper, fleqn]{article}
\usepackage{cp2425t}
\makeindex

%================= lhs2tex=====================================================%
%% ODER: format ==         = "\mathrel{==}"
%% ODER: format /=         = "\neq "
%
%
\makeatletter
\@ifundefined{lhs2tex.lhs2tex.sty.read}%
  {\@namedef{lhs2tex.lhs2tex.sty.read}{}%
   \newcommand\SkipToFmtEnd{}%
   \newcommand\EndFmtInput{}%
   \long\def\SkipToFmtEnd#1\EndFmtInput{}%
  }\SkipToFmtEnd

\newcommand\ReadOnlyOnce[1]{\@ifundefined{#1}{\@namedef{#1}{}}\SkipToFmtEnd}
\usepackage{amstext}
\usepackage{amssymb}
\usepackage{stmaryrd}
\DeclareFontFamily{OT1}{cmtex}{}
\DeclareFontShape{OT1}{cmtex}{m}{n}
  {<5><6><7><8>cmtex8
   <9>cmtex9
   <10><10.95><12><14.4><17.28><20.74><24.88>cmtex10}{}
\DeclareFontShape{OT1}{cmtex}{m}{it}
  {<-> ssub * cmtt/m/it}{}
\newcommand{\texfamily}{\fontfamily{cmtex}\selectfont}
\DeclareFontShape{OT1}{cmtt}{bx}{n}
  {<5><6><7><8>cmtt8
   <9>cmbtt9
   <10><10.95><12><14.4><17.28><20.74><24.88>cmbtt10}{}
\DeclareFontShape{OT1}{cmtex}{bx}{n}
  {<-> ssub * cmtt/bx/n}{}
\newcommand{\tex}[1]{\text{\texfamily#1}}	% NEU

\newcommand{\Sp}{\hskip.33334em\relax}


\newcommand{\Conid}[1]{\mathit{#1}}
\newcommand{\Varid}[1]{\mathit{#1}}
\newcommand{\anonymous}{\kern0.06em \vbox{\hrule\@width.5em}}
\newcommand{\plus}{\mathbin{+\!\!\!+}}
\newcommand{\bind}{\mathbin{>\!\!\!>\mkern-6.7mu=}}
\newcommand{\rbind}{\mathbin{=\mkern-6.7mu<\!\!\!<}}% suggested by Neil Mitchell
\newcommand{\sequ}{\mathbin{>\!\!\!>}}
\renewcommand{\leq}{\leqslant}
\renewcommand{\geq}{\geqslant}
\usepackage{polytable}

%mathindent has to be defined
\@ifundefined{mathindent}%
  {\newdimen\mathindent\mathindent\leftmargini}%
  {}%

\def\resethooks{%
  \global\let\SaveRestoreHook\empty
  \global\let\ColumnHook\empty}
\newcommand*{\savecolumns}[1][default]%
  {\g@addto@macro\SaveRestoreHook{\savecolumns[#1]}}
\newcommand*{\restorecolumns}[1][default]%
  {\g@addto@macro\SaveRestoreHook{\restorecolumns[#1]}}
\newcommand*{\aligncolumn}[2]%
  {\g@addto@macro\ColumnHook{\column{#1}{#2}}}

\resethooks

\newcommand{\onelinecommentchars}{\quad-{}- }
\newcommand{\commentbeginchars}{\enskip\{-}
\newcommand{\commentendchars}{-\}\enskip}

\newcommand{\visiblecomments}{%
  \let\onelinecomment=\onelinecommentchars
  \let\commentbegin=\commentbeginchars
  \let\commentend=\commentendchars}

\newcommand{\invisiblecomments}{%
  \let\onelinecomment=\empty
  \let\commentbegin=\empty
  \let\commentend=\empty}

\visiblecomments

\newlength{\blanklineskip}
\setlength{\blanklineskip}{0.66084ex}

\newcommand{\hsindent}[1]{\quad}% default is fixed indentation
\let\hspre\empty
\let\hspost\empty
\newcommand{\NB}{\textbf{NB}}
\newcommand{\Todo}[1]{$\langle$\textbf{To do:}~#1$\rangle$}

\EndFmtInput
\makeatother
%
%
%
%
%
%
% This package provides two environments suitable to take the place
% of hscode, called "plainhscode" and "arrayhscode". 
%
% The plain environment surrounds each code block by vertical space,
% and it uses \abovedisplayskip and \belowdisplayskip to get spacing
% similar to formulas. Note that if these dimensions are changed,
% the spacing around displayed math formulas changes as well.
% All code is indented using \leftskip.
%
% Changed 19.08.2004 to reflect changes in colorcode. Should work with
% CodeGroup.sty.
%
\ReadOnlyOnce{polycode.fmt}%
\makeatletter

\newcommand{\hsnewpar}[1]%
  {{\parskip=0pt\parindent=0pt\par\vskip #1\noindent}}

% can be used, for instance, to redefine the code size, by setting the
% command to \small or something alike
\newcommand{\hscodestyle}{}

% The command \sethscode can be used to switch the code formatting
% behaviour by mapping the hscode environment in the subst directive
% to a new LaTeX environment.

\newcommand{\sethscode}[1]%
  {\expandafter\let\expandafter\hscode\csname #1\endcsname
   \expandafter\let\expandafter\endhscode\csname end#1\endcsname}

% "compatibility" mode restores the non-polycode.fmt layout.

\newenvironment{compathscode}%
  {\par\noindent
   \advance\leftskip\mathindent
   \hscodestyle
   \let\\=\@normalcr
   \let\hspre\(\let\hspost\)%
   \pboxed}%
  {\endpboxed\)%
   \par\noindent
   \ignorespacesafterend}

\newcommand{\compaths}{\sethscode{compathscode}}

% "plain" mode is the proposed default.
% It should now work with \centering.
% This required some changes. The old version
% is still available for reference as oldplainhscode.

\newenvironment{plainhscode}%
  {\hsnewpar\abovedisplayskip
   \advance\leftskip\mathindent
   \hscodestyle
   \let\hspre\(\let\hspost\)%
   \pboxed}%
  {\endpboxed%
   \hsnewpar\belowdisplayskip
   \ignorespacesafterend}

\newenvironment{oldplainhscode}%
  {\hsnewpar\abovedisplayskip
   \advance\leftskip\mathindent
   \hscodestyle
   \let\\=\@normalcr
   \(\pboxed}%
  {\endpboxed\)%
   \hsnewpar\belowdisplayskip
   \ignorespacesafterend}

% Here, we make plainhscode the default environment.

\newcommand{\plainhs}{\sethscode{plainhscode}}
\newcommand{\oldplainhs}{\sethscode{oldplainhscode}}
\plainhs

% The arrayhscode is like plain, but makes use of polytable's
% parray environment which disallows page breaks in code blocks.

\newenvironment{arrayhscode}%
  {\hsnewpar\abovedisplayskip
   \advance\leftskip\mathindent
   \hscodestyle
   \let\\=\@normalcr
   \(\parray}%
  {\endparray\)%
   \hsnewpar\belowdisplayskip
   \ignorespacesafterend}

\newcommand{\arrayhs}{\sethscode{arrayhscode}}

% The mathhscode environment also makes use of polytable's parray 
% environment. It is supposed to be used only inside math mode 
% (I used it to typeset the type rules in my thesis).

\newenvironment{mathhscode}%
  {\parray}{\endparray}

\newcommand{\mathhs}{\sethscode{mathhscode}}

% texths is similar to mathhs, but works in text mode.

\newenvironment{texthscode}%
  {\(\parray}{\endparray\)}

\newcommand{\texths}{\sethscode{texthscode}}

% The framed environment places code in a framed box.

\def\codeframewidth{\arrayrulewidth}
\RequirePackage{calc}

\newenvironment{framedhscode}%
  {\parskip=\abovedisplayskip\par\noindent
   \hscodestyle
   \arrayrulewidth=\codeframewidth
   \tabular{@{}|p{\linewidth-2\arraycolsep-2\arrayrulewidth-2pt}|@{}}%
   \hline\framedhslinecorrect\\{-1.5ex}%
   \let\endoflinesave=\\
   \let\\=\@normalcr
   \(\pboxed}%
  {\endpboxed\)%
   \framedhslinecorrect\endoflinesave{.5ex}\hline
   \endtabular
   \parskip=\belowdisplayskip\par\noindent
   \ignorespacesafterend}

\newcommand{\framedhslinecorrect}[2]%
  {#1[#2]}

\newcommand{\framedhs}{\sethscode{framedhscode}}

% The inlinehscode environment is an experimental environment
% that can be used to typeset displayed code inline.

\newenvironment{inlinehscode}%
  {\(\def\column##1##2{}%
   \let\>\undefined\let\<\undefined\let\\\undefined
   \newcommand\>[1][]{}\newcommand\<[1][]{}\newcommand\\[1][]{}%
   \def\fromto##1##2##3{##3}%
   \def\nextline{}}{\) }%

\newcommand{\inlinehs}{\sethscode{inlinehscode}}

% The joincode environment is a separate environment that
% can be used to surround and thereby connect multiple code
% blocks.

\newenvironment{joincode}%
  {\let\orighscode=\hscode
   \let\origendhscode=\endhscode
   \def\endhscode{\def\hscode{\endgroup\def\@currenvir{hscode}\\}\begingroup}
   %\let\SaveRestoreHook=\empty
   %\let\ColumnHook=\empty
   %\let\resethooks=\empty
   \orighscode\def\hscode{\endgroup\def\@currenvir{hscode}}}%
  {\origendhscode
   \global\let\hscode=\orighscode
   \global\let\endhscode=\origendhscode}%

\makeatother
\EndFmtInput
%


%------------------------------------------------------------------------------%


%====== DEFINIR GRUPO E ELEMENTOS =============================================%

\group{G2}
\studentA{104356}{João d'Araújo Dias Lobo }
\studentB{90817}{Mariana Rocha Cristino }
\studentC{104439}{Rita da Cunha Camacho }

%==============================================================================%

\begin{document}

\sffamily
\setlength{\parindent}{0em}
\emergencystretch 3em
\renewcommand{\baselinestretch}{1.25} 
\input{Cover}
\pagestyle{pagestyle}
\setlength{\parindent}{1em}
\newgeometry{left=25mm,right=20mm,top=25mm,bottom=25mm}

\section*{Preâmbulo}

Em \CP\ pretende-se ensinar a progra\-mação de computadores como uma disciplina
científica. Para isso parte-se de um repertório de \emph{combinadores} que
formam uma álgebra da programação % (conjunto de leis universais e seus corolários)
e usam-se esses combinadores para construir programas \emph{composicionalmente},
isto é, agregando programas já existentes.

Na sequência pedagógica dos planos de estudo dos cursos que têm esta disciplina,
opta-se pela aplicação deste método à programação em \Haskell\ (sem prejuízo
da sua aplicação a outras linguagens funcionais). Assim, o presente trabalho
prático coloca os alunos perante problemas concretos que deverão ser implementados
em \Haskell. Há ainda um outro objectivo: o de ensinar a documentar programas,
a validá-los e a produzir textos técnico-científicos de qualidade.

Antes de abordarem os problemas propostos no trabalho, os grupos devem ler
com atenção o anexo \ref{sec:documentacao} onde encontrarão as instruções
relativas ao \emph{software} a instalar, etc.

Valoriza-se a escrita de \emph{pouco} código que corresponda a soluções simples
e elegantes que utilizem os combinadores de ordem superior estudados na disciplina.


\Problema

Esta questão aborda um problema que é conhecido pela designação '\emph{H-index of a Histogram}'
e que se formula facilmente:
\begin{quote}\em
O h-index de um histograma é o maior número \ensuremath{\Varid{n}} de barras do histograma cuja altura é maior ou igual a \ensuremath{\Varid{n}}.
\end{quote}
Por exemplo, o histograma 
\begin{hscode}\SaveRestoreHook
\column{B}{@{}>{\hspre}l<{\hspost}@{}}%
\column{E}{@{}>{\hspre}l<{\hspost}@{}}%
\>[B]{}\Varid{h}\mathrel{=}[\mskip1.5mu \mathrm{5},\mathrm{2},\mathrm{7},\mathrm{1},\mathrm{8},\mathrm{6},\mathrm{4},\mathrm{9}\mskip1.5mu]{}\<[E]%
\ColumnHook
\end{hscode}\resethooks
que se mostra na figura
	\histograma
tem \ensuremath{\Varid{hindex}\;\Varid{h}\mathrel{=}\mathrm{5}}
pois há \ensuremath{\mathrm{5}} colunas maiores que \ensuremath{\mathrm{5}}. (Não é \ensuremath{\mathrm{6}} pois maiores ou iguais que seis só há quatro.)

Pretende-se definida como um catamorfismo, anamorfismo ou hilomorfismo uma função em Haskell
\begin{hscode}\SaveRestoreHook
\column{B}{@{}>{\hspre}l<{\hspost}@{}}%
\column{E}{@{}>{\hspre}l<{\hspost}@{}}%
\>[B]{}\Varid{hindex}\mathbin{::}[\mskip1.5mu \Conid{Int}\mskip1.5mu]\to (\Conid{Int},[\mskip1.5mu \Conid{Int}\mskip1.5mu]){}\<[E]%
\ColumnHook
\end{hscode}\resethooks
tal que, para \ensuremath{(\Varid{i},\Varid{x})\mathrel{=}\Varid{hindex}\;\Varid{h}}, \ensuremath{\Varid{i}} é o H-index de \ensuremath{\Varid{h}} e \ensuremath{\Varid{x}} é a lista de colunas de \ensuremath{\Varid{h}} que para ele contribuem.

A proposta de \ensuremath{\Varid{hindex}} deverá vir acompanhada de um \textbf{diagrama} ilustrativo.

\Problema

Pelo \href{https://en.wikipedia.org/wiki/Fundamental_theorem_of_arithmetic}{teorema
fundamental da aritmética}, todo número inteiro positivo tem uma única factorização
prima.  For exemplo,
\begin{tabbing}\ttfamily
~primes~455\\
\ttfamily ~\char91{}5\char44{}7\char44{}13\char93{}\\
\ttfamily ~primes~433\\
\ttfamily ~\char91{}433\char93{}\\
\ttfamily ~primes~230\\
\ttfamily ~\char91{}2\char44{}5\char44{}23\char93{}
\end{tabbing}

\begin{enumerate}

\item	
Implemente como anamorfismo de listas a função
\begin{hscode}\SaveRestoreHook
\column{B}{@{}>{\hspre}l<{\hspost}@{}}%
\column{E}{@{}>{\hspre}l<{\hspost}@{}}%
\>[B]{}\Varid{primes}\mathbin{::}\mathbb{Z}\to [\mskip1.5mu \mathbb{Z}\mskip1.5mu]{}\<[E]%
\ColumnHook
\end{hscode}\resethooks
que deverá, recebendo um número inteiro positivo, devolver a respectiva lista
de factores primos.

A proposta de \ensuremath{\Varid{primes}} deverá vir acompanhada de um \textbf{diagrama} ilustrativo.

\item A figura mostra a ``\emph{árvore dos primos}'' dos números \ensuremath{[\mskip1.5mu \mathrm{455},\mathrm{669},\mathrm{6645},\mathrm{34},\mathrm{12},\mathrm{2}\mskip1.5mu]}.

      \primes

Com base na alínea anterior, implemente uma função em Haskell que faça a
geração de uma tal árvore a partir de uma lista de inteiros:

\begin{hscode}\SaveRestoreHook
\column{B}{@{}>{\hspre}l<{\hspost}@{}}%
\column{E}{@{}>{\hspre}l<{\hspost}@{}}%
\>[B]{}\Varid{prime\char95 tree}\mathbin{::}[\mskip1.5mu \mathbb{Z}\mskip1.5mu]\to \Conid{Exp}\;\mathbb{Z}\;\mathbb{Z}{}\<[E]%
\ColumnHook
\end{hscode}\resethooks

\textbf{Sugestão}: escreva o mínimo de código possível em \ensuremath{\Varid{prime\char95 tree}} investigando
cuidadosamente que funções disponíveis nas bibliotecas que são dadas podem
ser reutilizadas.%
\footnote{Pense sempre na sua produtividade quando está a programar --- essa
atitude será valorizada por qualquer empregador que vier a ter.}

\end{enumerate}

\Problema

A convolução \ensuremath{\Varid{a}\star \Varid{b}} de duas listas \ensuremath{\Varid{a}} e \ensuremath{\Varid{b}} --- uma operação relevante em computação
---  está muito bem explicada
\href{https://www.youtube.com/watch?v=KuXjwB4LzSA}{neste vídeo} do canal
\textbf{3Blue1Brown} do YouTube,
a partir de \href{https://www.youtube.com/watch?v=KuXjwB4LzSA&t=390s}{\ensuremath{\Varid{t}\mathrel{=}\mathrm{6}\mathbin{:}\mathrm{30}}}.
Aí se mostra como, por exemplo:
\begin{quote}
\ensuremath{[\mskip1.5mu \mathrm{1},\mathrm{2},\mathrm{3}\mskip1.5mu]\star [\mskip1.5mu \mathrm{4},\mathrm{5},\mathrm{6}\mskip1.5mu]\mathrel{=}[\mskip1.5mu \mathrm{4},\mathrm{13},\mathrm{28},\mathrm{27},\mathrm{18}\mskip1.5mu]} 
\end{quote}
A solução abaixo, proposta pelo chatGPT,
\begin{hscode}\SaveRestoreHook
\column{B}{@{}>{\hspre}l<{\hspost}@{}}%
\column{3}{@{}>{\hspre}l<{\hspost}@{}}%
\column{E}{@{}>{\hspre}l<{\hspost}@{}}%
\>[B]{}\Varid{convolve}\mathbin{::}\Conid{Num}\;\Varid{a}\Rightarrow [\mskip1.5mu \Varid{a}\mskip1.5mu]\to [\mskip1.5mu \Varid{a}\mskip1.5mu]\to [\mskip1.5mu \Varid{a}\mskip1.5mu]{}\<[E]%
\\
\>[B]{}\Varid{convolve}\;\Varid{xs}\;\Varid{ys}\mathrel{=}[\mskip1.5mu \Varid{sum}\mathbin{\$}\Varid{zipWith}\;(\mathbin{*})\;(\Varid{take}\;\Varid{n}\;(\Varid{drop}\;\Varid{i}\;\Varid{xs}))\;\Varid{ys}\mid \Varid{i}\leftarrow [\mskip1.5mu \mathrm{0}\mathinner{\ldotp\ldotp}(\length \;\Varid{xs}\mathbin{-}\Varid{n})\mskip1.5mu]\mskip1.5mu]{}\<[E]%
\\
\>[B]{}\hsindent{3}{}\<[3]%
\>[3]{}\mathbf{where}\;\Varid{n}\mathrel{=}\length \;\Varid{ys}{}\<[E]%
\ColumnHook
\end{hscode}\resethooks
está manifestamente errada, pois \ensuremath{\Varid{convolve}\;[\mskip1.5mu \mathrm{1},\mathrm{2},\mathrm{3}\mskip1.5mu]\;[\mskip1.5mu \mathrm{4},\mathrm{5},\mathrm{6}\mskip1.5mu]\mathrel{=}[\mskip1.5mu \mathrm{32}\mskip1.5mu]} (!).

Proponha, explicando-a devidamente, uma solução sua para \ensuremath{\Varid{convolve}}.
Valorizar-se-á a economia de código e o recurso aos combinadores \emph{pointfree} estudados
na disciplina, em particular a triologia \emph{ana-cata-hilo} de tipos disponíveis nas
bibliotecas dadas ou a definir.

\Problema

Considere-se a seguinte sintaxe (abstrata e simplificada) para \textbf{expressões numéricas} (em \ensuremath{\Varid{b}}) com variáveis (em \ensuremath{\Varid{a}}),
\begin{hscode}\SaveRestoreHook
\column{B}{@{}>{\hspre}l<{\hspost}@{}}%
\column{19}{@{}>{\hspre}l<{\hspost}@{}}%
\column{50}{@{}>{\hspre}l<{\hspost}@{}}%
\column{E}{@{}>{\hspre}l<{\hspost}@{}}%
\>[B]{}\mathbf{data}\;\Conid{Expr}\;\Varid{b}\;\Varid{a}\mathrel{=}{}\<[19]%
\>[19]{}\Conid{V}\;\Varid{a}\mid \Conid{N}\;\Varid{b}\mid \Conid{T}\;\Conid{Op}\;[\mskip1.5mu \Conid{Expr}\;\Varid{b}\;\Varid{a}\mskip1.5mu]\;{}\<[50]%
\>[50]{}\mathbf{deriving}\;(\Conid{Show},\Conid{Eq}){}\<[E]%
\\[\blanklineskip]%
\>[B]{}\mathbf{data}\;\Conid{Op}\mathrel{=}\Conid{ITE}\mid \Conid{Add}\mid \Conid{Mul}\mid \Conid{Suc}\;\mathbf{deriving}\;(\Conid{Show},\Conid{Eq}){}\<[E]%
\ColumnHook
\end{hscode}\resethooks
possivelmente condicionais (cf.\ \ensuremath{\Conid{ITE}}, i.e.\ o operador condicional ``if-then-else``).
Por exemplo, a árvore mostrada a seguir
        \treeA
representa a expressão
\begin{eqnarray}
        \ensuremath{\Varid{ite}\;(\Conid{V}\;\text{\ttfamily \char34 x\char34})\;(\Conid{N}\;\mathrm{0})\;(\Varid{multi}\;(\Conid{V}\;\text{\ttfamily \char34 y\char34})\;(\Varid{soma}\;(\Conid{N}\;\mathrm{3})\;(\Conid{V}\;\text{\ttfamily \char34 y\char34})))}
        \label{eq:expr}
\end{eqnarray}
--- i.e.\ \ensuremath{\mathbf{if}\;\Varid{x}\;\mathbf{then}\;\mathrm{0}\;\mathbf{else}\;\Varid{y}\mathbin{*}(\mathrm{3}\mathbin{+}\Varid{y})} ---
assumindo as ``helper functions'':
\begin{hscode}\SaveRestoreHook
\column{B}{@{}>{\hspre}l<{\hspost}@{}}%
\column{7}{@{}>{\hspre}l<{\hspost}@{}}%
\column{E}{@{}>{\hspre}l<{\hspost}@{}}%
\>[B]{}\Varid{soma}\;{}\<[7]%
\>[7]{}\Varid{x}\;\Varid{y}\mathrel{=}\Conid{T}\;\Conid{Add}\;[\mskip1.5mu \Varid{x},\Varid{y}\mskip1.5mu]{}\<[E]%
\\
\>[B]{}\Varid{multi}\;\Varid{x}\;\Varid{y}\mathrel{=}\Conid{T}\;\Conid{Mul}\;[\mskip1.5mu \Varid{x},\Varid{y}\mskip1.5mu]{}\<[E]%
\\
\>[B]{}\Varid{ite}\;\Varid{x}\;\Varid{y}\;\Varid{z}\mathrel{=}\Conid{T}\;\Conid{ITE}\;[\mskip1.5mu \Varid{x},\Varid{y},\Varid{z}\mskip1.5mu]{}\<[E]%
\ColumnHook
\end{hscode}\resethooks

No anexo \ref{sec:codigo} propôe-se uma base para o tipo Expr (\ensuremath{\Varid{baseExpr}}) e a 
correspondente algebra \ensuremath{\Varid{inExpr}} para construção do tipo \ensuremath{\Conid{Expr}}.

\begin{enumerate}
\item        Complete as restantes definições da biblioteca \ensuremath{\Conid{Expr}}  pedidas no anexo \ref{sec:resolucao}.
\item        No mesmo anexo, declare \ensuremath{\Conid{Expr}\;\Varid{b}} como instância da classe \ensuremath{\Conid{Monad}}. \textbf{Sugestão}: relembre os exercícios da ficha 12.
\item        Defina como um catamorfismo de \ensuremath{\Conid{Expr}} a sua versão monádia, que deverá ter o tipo:
\begin{hscode}\SaveRestoreHook
\column{B}{@{}>{\hspre}l<{\hspost}@{}}%
\column{E}{@{}>{\hspre}l<{\hspost}@{}}%
\>[B]{}\Varid{mcataExpr}\mathbin{::}\Conid{Monad}\;\Varid{m}\Rightarrow (\Varid{a}+(\Varid{b}+(\Conid{Op},\Varid{m}\;[\mskip1.5mu \Varid{c}\mskip1.5mu]))\to \Varid{m}\;\Varid{c})\to \Conid{Expr}\;\Varid{b}\;\Varid{a}\to \Varid{m}\;\Varid{c}{}\<[E]%
\ColumnHook
\end{hscode}\resethooks
\item        Para se avaliar uma expressão é preciso que todas as suas variáveis estejam instanciadas.
Complete a definição da função
\begin{hscode}\SaveRestoreHook
\column{B}{@{}>{\hspre}l<{\hspost}@{}}%
\column{E}{@{}>{\hspre}l<{\hspost}@{}}%
\>[B]{}\Varid{let\char95 exp}\mathbin{::}(\Conid{Num}\;\Varid{c})\Rightarrow (\Varid{a}\to \Conid{Expr}\;\Varid{c}\;\Varid{b})\to \Conid{Expr}\;\Varid{c}\;\Varid{a}\to \Conid{Expr}\;\Varid{c}\;\Varid{b}{}\<[E]%
\ColumnHook
\end{hscode}\resethooks
que, dada uma expressão com variáveis em \ensuremath{\Varid{a}} e uma função que a cada uma dessas variáveis atribui uma
expressão (\ensuremath{\Varid{a}\to \Conid{Expr}\;\Varid{c}\;\Varid{b}}), faz a correspondente substituição.\footnote{Cf.\ expressões \ensuremath{\mathbf{let}\mathbin{...}\mathbf{in}\mathbin{...}}.}
Por exemplo, dada
\begin{hscode}\SaveRestoreHook
\column{B}{@{}>{\hspre}l<{\hspost}@{}}%
\column{7}{@{}>{\hspre}l<{\hspost}@{}}%
\column{E}{@{}>{\hspre}l<{\hspost}@{}}%
\>[B]{}\Varid{f}\;\text{\ttfamily \char34 x\char34}\mathrel{=}\Conid{N}\;\mathrm{0}{}\<[E]%
\\
\>[B]{}\Varid{f}\;\text{\ttfamily \char34 y\char34}\mathrel{=}\Conid{N}\;\mathrm{5}{}\<[E]%
\\
\>[B]{}\Varid{f}\;\anonymous {}\<[7]%
\>[7]{}\mathrel{=}\Conid{N}\;\mathrm{99}{}\<[E]%
\ColumnHook
\end{hscode}\resethooks
ter-se-á
\begin{hscode}\SaveRestoreHook
\column{B}{@{}>{\hspre}l<{\hspost}@{}}%
\column{9}{@{}>{\hspre}l<{\hspost}@{}}%
\column{E}{@{}>{\hspre}l<{\hspost}@{}}%
\>[9]{}\Varid{let\char95 exp}\;\Varid{f}\;\Varid{e}\mathrel{=}\Conid{T}\;\Conid{ITE}\;[\mskip1.5mu \Conid{N}\;\mathrm{1},\Conid{N}\;\mathrm{0},\Conid{T}\;\Conid{Mul}\;[\mskip1.5mu \Conid{N}\;\mathrm{5},\Conid{T}\;\Conid{Add}\;[\mskip1.5mu \Conid{N}\;\mathrm{3},\Conid{N}\;\mathrm{1}\mskip1.5mu]\mskip1.5mu]\mskip1.5mu]{}\<[E]%
\ColumnHook
\end{hscode}\resethooks
isto é, a árvore da figura a seguir: 
        \treeB

\item Finalmente, defina a função de avaliação de uma expressão, com tipo

\begin{hscode}\SaveRestoreHook
\column{B}{@{}>{\hspre}l<{\hspost}@{}}%
\column{32}{@{}>{\hspre}l<{\hspost}@{}}%
\column{E}{@{}>{\hspre}l<{\hspost}@{}}%
\>[B]{}\Varid{evaluate}\mathbin{::}(\Conid{Num}\;\Varid{a},\Conid{Ord}\;\Varid{a})\Rightarrow {}\<[32]%
\>[32]{}\Conid{Expr}\;\Varid{a}\;\Varid{b}\to \Conid{Maybe}\;\Varid{a}{}\<[E]%
\ColumnHook
\end{hscode}\resethooks

que deverá ter em conta as seguintes situações de erro:

\begin{enumerate}

\item \emph{Variáveis} --- para ser avaliada, \ensuremath{\Varid{x}} em \ensuremath{\Varid{evaluate}\;\Varid{x}} não pode conter variáveis. Assim, por exemplo,
        \begin{hscode}\SaveRestoreHook
\column{B}{@{}>{\hspre}l<{\hspost}@{}}%
\column{9}{@{}>{\hspre}l<{\hspost}@{}}%
\column{E}{@{}>{\hspre}l<{\hspost}@{}}%
\>[9]{}\Varid{evaluate}\;\Varid{e}\mathrel{=}\Conid{Nothing}{}\<[E]%
\\
\>[9]{}\Varid{evaluate}\;(\Varid{let\char95 exp}\;\Varid{f}\;\Varid{e})\mathrel{=}\Conid{Just}\;\mathrm{40}{}\<[E]%
\ColumnHook
\end{hscode}\resethooks
para \ensuremath{\Varid{f}} e \ensuremath{\Varid{e}}  dadas acima.

\item \emph{Aridades} --- todas as ocorrências dos operadores deverão ter
      o devido número de sub-expressões, por exemplo:
        \begin{hscode}\SaveRestoreHook
\column{B}{@{}>{\hspre}l<{\hspost}@{}}%
\column{9}{@{}>{\hspre}l<{\hspost}@{}}%
\column{E}{@{}>{\hspre}l<{\hspost}@{}}%
\>[9]{}\Varid{evaluate}\;(\Conid{T}\;\Conid{Add}\;[\mskip1.5mu \Conid{N}\;\mathrm{2},\Conid{N}\;\mathrm{3}\mskip1.5mu])\mathrel{=}\Conid{Just}\;\mathrm{5}{}\<[E]%
\\
\>[9]{}\Varid{evaluate}\;(\Conid{T}\;\Conid{Mul}\;[\mskip1.5mu \Conid{N}\;\mathrm{2}\mskip1.5mu])\mathrel{=}\Conid{Nothing}{}\<[E]%
\ColumnHook
\end{hscode}\resethooks

\end{enumerate}

\end{enumerate}

\noindent
\textbf{Sugestão}: de novo se insiste na escrita do mínimo de código possível,
tirando partido da riqueza estrutural do tipo \ensuremath{\Conid{Expr}} que é assunto desta questão.
Sugere-se também o recurso a diagramas para explicar as soluções propostas.

\part*{Anexos}

\appendix

\section{Natureza do trabalho a realizar}
\label{sec:documentacao}
Este trabalho teórico-prático deve ser realizado por grupos de 3 alunos.
Os detalhes da avaliação (datas para submissão do relatório e sua defesa
oral) são os que forem publicados na \cp{página da disciplina} na \emph{internet}.

Recomenda-se uma abordagem participativa dos membros do grupo em \textbf{todos}
os exercícios do trabalho, para assim poderem responder a qualquer questão
colocada na \emph{defesa oral} do relatório.

Para cumprir de forma integrada os objectivos do trabalho vamos recorrer
a uma técnica de programa\-ção dita ``\litp{literária}'' \cite{Kn92}, cujo
princípio base é o seguinte:
%
\begin{quote}\em
	Um programa e a sua documentação devem coincidir.
\end{quote}
%
Por outras palavras, o \textbf{código fonte} e a \textbf{documentação} de um
programa deverão estar no mesmo ficheiro.

O ficheiro \texttt{cp2425t.pdf} que está a ler é já um exemplo de
\litp{programação literária}: foi gerado a partir do texto fonte
\texttt{cp2425t.lhs}\footnote{O sufixo `lhs' quer dizer
\emph{\lhaskell{literate Haskell}}.} que encontrará no \MaterialPedagogico\
desta disciplina des\-com\-pactando o ficheiro \texttt{cp2425t.zip}.

Como se mostra no esquema abaixo, de um único ficheiro (\ensuremath{\Varid{lhs}})
gera-se um PDF ou faz-se a interpretação do código \Haskell\ que ele inclui:

	\esquema

Vê-se assim que, para além do \GHCi, serão necessários os executáveis \PdfLatex\ e
\LhsToTeX. Para facilitar a instalação e evitar problemas de versões e
conflitos com sistemas operativos, é recomendado o uso do \Docker\ tal como
a seguir se descreve.

\section{Docker} \label{sec:docker}

Recomenda-se o uso do \container\ cuja imagem é gerada pelo \Docker\ a partir do ficheiro
\texttt{Dockerfile} que se encontra na diretoria que resulta de descompactar
\texttt{cp2425t.zip}. Este \container\ deverá ser usado na execução
do \GHCi\ e dos comandos relativos ao \Latex. (Ver também a \texttt{Makefile}
que é disponibilizada.)

Após \href{https://docs.docker.com/engine/install/}{instalar o Docker} e
descarregar o referido zip com o código fonte do trabalho,
basta executar os seguintes comandos:
\begin{Verbatim}[fontsize=\small]
    $ docker build -t cp2425t .
    $ docker run -v ${PWD}:/cp2425t -it cp2425t
\end{Verbatim}
\textbf{NB}: O objetivo é que o container\ seja usado \emph{apenas} 
para executar o \GHCi\ e os comandos relativos ao \Latex.
Deste modo, é criado um \textit{volume} (cf.\ a opção \texttt{-v \$\{PWD\}:/cp2425t}) 
que permite que a diretoria em que se encontra na sua máquina local 
e a diretoria \texttt{/cp2425t} no \container\ sejam partilhadas.

Pretende-se então que visualize/edite os ficheiros na sua máquina local e que
os compile no \container, executando:
\begin{Verbatim}[fontsize=\small]
    $ lhs2TeX cp2425t.lhs > cp2425t.tex
    $ pdflatex cp2425t
\end{Verbatim}
\LhsToTeX\ é o pre-processador que faz ``pretty printing'' de código Haskell
em \Latex\ e que faz parte já do \container. Alternativamente, basta executar
\begin{Verbatim}[fontsize=\small]
    $ make
\end{Verbatim}
para obter o mesmo efeito que acima.

Por outro lado, o mesmo ficheiro \texttt{cp2425t.lhs} é executável e contém
o ``kit'' básico, escrito em \Haskell, para realizar o trabalho. Basta executar
\begin{Verbatim}[fontsize=\small]
    $ ghci cp2425t.lhs
\end{Verbatim}

\noindent Abra o ficheiro \texttt{cp2425t.lhs} no seu editor de texto preferido
e verifique que assim é: todo o texto que se encontra dentro do ambiente
\begin{quote}\small\tt
\text{\ttfamily \char92{}begin\char123{}code\char125{}}
\\ ... \\
\text{\ttfamily \char92{}end\char123{}code\char125{}}
\end{quote}
é seleccionado pelo \GHCi\ para ser executado.

\section{Em que consiste o TP}

Em que consiste, então, o \emph{relatório} a que se referiu acima?
É a edição do texto que está a ser lido, preenchendo o anexo \ref{sec:resolucao}
com as respostas. O relatório deverá conter ainda a identificação dos membros
do grupo de trabalho, no local respectivo da folha de rosto.

Para gerar o PDF integral do relatório deve-se ainda correr os comando seguintes,
que actualizam a bibliografia (com \Bibtex) e o índice remissivo (com \Makeindex),
\begin{Verbatim}[fontsize=\small]
    $ bibtex cp2425t.aux
    $ makeindex cp2425t.idx
\end{Verbatim}
e recompilar o texto como acima se indicou. (Como já se disse, pode fazê-lo
correndo simplesmente \texttt{make} no \container.)

No anexo \ref{sec:codigo} disponibiliza-se algum código \Haskell\ relativo
aos problemas que são colocados. Esse anexo deverá ser consultado e analisado
à medida que isso for necessário.

Deve ser feito uso da \litp{programação literária} para documentar bem o código que se
desenvolver, em particular fazendo diagramas explicativos do que foi feito e
tal como se explica no anexo \ref{sec:diagramas} que se segue.

\section{Como exprimir cálculos e diagramas em LaTeX/lhs2TeX} \label{sec:diagramas}

Como primeiro exemplo, estudar o texto fonte (\lhstotex{lhs}) do que está a ler\footnote{
Procure e.g.\ por \texttt{"sec:diagramas"}.} onde se obtém o efeito seguinte:\footnote{Exemplos
tirados de \cite{Ol18}.}
\begin{eqnarray*}
\start
\ensuremath{\Varid{id}\mathrel{=}\conj{\Varid{f}}{\Varid{g}}}
\just\equiv{ universal property }
\ensuremath{\begin{lcbr}\p1\comp \Varid{id}\mathrel{=}\Varid{f}\\\p2\comp \Varid{id}\mathrel{=}\Varid{g}\end{lcbr}}
\just\equiv{ identity }
\ensuremath{\begin{lcbr}\p1\mathrel{=}\Varid{f}\\\p2\mathrel{=}\Varid{g}\end{lcbr}}
\qed
\end{eqnarray*}

Os diagramas podem ser produzidos recorrendo à \emph{package} \Xymatrix, por exemplo:
\begin{eqnarray*}
\xymatrix@C=2cm{
    \ensuremath{\N_0}
           \ar[d]_-{\ensuremath{\cataNat{\Varid{g}}}}
&
    \ensuremath{\mathrm{1}\mathbin{+}\N_0}
           \ar[d]^{\ensuremath{\Varid{id}\mathbin{+}\cataNat{\Varid{g}}}}
           \ar[l]_-{\ensuremath{\mathsf{in}}}
\\
     \ensuremath{\Conid{B}}
&
     \ensuremath{\mathrm{1}\mathbin{+}\Conid{B}}
           \ar[l]^-{\ensuremath{\Varid{g}}}
}
\end{eqnarray*}

\section{Código fornecido}\label{sec:codigo}

\subsection*{Problema 1}

\begin{hscode}\SaveRestoreHook
\column{B}{@{}>{\hspre}l<{\hspost}@{}}%
\column{E}{@{}>{\hspre}l<{\hspost}@{}}%
\>[B]{}\Varid{h}\mathbin{::}[\mskip1.5mu \Conid{Int}\mskip1.5mu]{}\<[E]%
\ColumnHook
\end{hscode}\resethooks

\subsection*{Problema 4}
Definição do tipo:
\begin{hscode}\SaveRestoreHook
\column{B}{@{}>{\hspre}l<{\hspost}@{}}%
\column{E}{@{}>{\hspre}l<{\hspost}@{}}%
\>[B]{}\Varid{inExpr}\mathrel{=}\alt{\Conid{V}}{\alt{\Conid{N}}{\uncurry{\Conid{T}}}}{}\<[E]%
\\[\blanklineskip]%
\>[B]{}\Varid{baseExpr}\;\Varid{g}\;\Varid{h}\;\Varid{f}\mathrel{=}\Varid{g}+(\Varid{h}+\Varid{id}\times\map \;\Varid{f}){}\<[E]%
\ColumnHook
\end{hscode}\resethooks
Exemplos de expressões:
\begin{hscode}\SaveRestoreHook
\column{B}{@{}>{\hspre}l<{\hspost}@{}}%
\column{E}{@{}>{\hspre}l<{\hspost}@{}}%
\>[B]{}\Varid{e}\mathrel{=}\Varid{ite}\;(\Conid{V}\;\text{\ttfamily \char34 x\char34})\;(\Conid{N}\;\mathrm{0})\;(\Varid{multi}\;(\Conid{V}\;\text{\ttfamily \char34 y\char34})\;(\Varid{soma}\;(\Conid{N}\;\mathrm{3})\;(\Conid{V}\;\text{\ttfamily \char34 y\char34}))){}\<[E]%
\\
\>[B]{}\Varid{i}\mathrel{=}\Varid{ite}\;(\Conid{V}\;\text{\ttfamily \char34 x\char34})\;(\Conid{N}\;\mathrm{1})\;(\Varid{multi}\;(\Conid{V}\;\text{\ttfamily \char34 y\char34})\;(\Varid{soma}\;(\Conid{N}\;(\mathrm{3}\mathbin{/}\mathrm{5}))\;(\Conid{V}\;\text{\ttfamily \char34 y\char34}))){}\<[E]%
\ColumnHook
\end{hscode}\resethooks
Exemplo de teste:
\begin{hscode}\SaveRestoreHook
\column{B}{@{}>{\hspre}l<{\hspost}@{}}%
\column{5}{@{}>{\hspre}l<{\hspost}@{}}%
\column{E}{@{}>{\hspre}l<{\hspost}@{}}%
\>[B]{}\Varid{teste}\mathrel{=}\Varid{evaluate}\;(\Varid{let\char95 exp}\;\Varid{f}\;\Varid{i})\equiv \Conid{Just}\;(\mathrm{26}\mathbin{/}\mathrm{245}){}\<[E]%
\\
\>[B]{}\hsindent{5}{}\<[5]%
\>[5]{}\mathbf{where}\;\Varid{f}\;\text{\ttfamily \char34 x\char34}\mathrel{=}\Conid{N}\;\mathrm{0};\Varid{f}\;\text{\ttfamily \char34 y\char34}\mathrel{=}\Conid{N}\;(\mathrm{1}\mathbin{/}\mathrm{7}){}\<[E]%
\ColumnHook
\end{hscode}\resethooks

%----------------- Soluções dos alunos -----------------------------------------%

\section{Soluções dos alunos}\label{sec:resolucao}
Os alunos devem colocar neste anexo as suas soluções para os exercícios
propostos, de acordo com o ``layout'' que se fornece.
Não podem ser alterados os nomes ou tipos das funções dadas, mas pode ser
adicionado texto ao anexo, bem como diagramas e/ou outras funções auxiliares
que sejam necessárias.

\noindent
\textbf{Importante}: Não pode ser alterado o texto deste ficheiro fora deste anexo.

\subsection*{Problema 1}

A função \ensuremath{\Varid{hindex}} foi implementada como um hilomorfismo de \ensuremath{\Conid{BTree}} (\ensuremath{\Varid{hyloBTree}}), visto que o problema assemelha-se ao processo de ordenação \ensuremath{\Varid{qsort}},
que também utiliza um hilomorfismo. A ideia principal foi usar a partição de elementos como o \ensuremath{\Varid{qSort}} usa e adaptar o restante processo para calcular o h-index.

A função \ensuremath{\Varid{hindex}} é representada pelo seguinte diagrama:

\begin{eqnarray*}
\xymatrix@C=2cm{
    \ensuremath{\mathbb{Z}}^*
           \ar[d]_-{\ensuremath{\ana{\Varid{qsep}}}}
           \ar[r]^-{\ensuremath{\Varid{qsep}}}
&
    \ensuremath{\mathrm{1}+(\mathbb{Z},(\mathbb{Z}}^*\ensuremath{,\mathbb{Z}}^*\ensuremath{))}
           \ar[d]^{\ensuremath{\Varid{id}\mathbin{+}\Varid{id}\times(\ana{\Varid{qsep}}\times\ana{\Varid{qsep}})}}
\\
    \ensuremath{\Conid{BTree}\;\mathbb{Z}}
           \ar[d]_-{\ensuremath{\cataNat{\Varid{f}}}}
           \ar@/_/[r]_-{\ensuremath{\Varid{out}}}
&
    \ensuremath{\mathrm{1}+(\mathbb{Z},(\Conid{BTree}\;\mathbb{Z},\Conid{BTree}\;\mathbb{Z}))}
           \ar[d]^{\ensuremath{\Varid{id}\mathbin{+}\Varid{id}\times(\llparenthesis\, \Varid{f}\,\rrparenthesis\times\llparenthesis\, \Varid{f}\,\rrparenthesis)}}
           \ar@/_/[l]_-{\ensuremath{\mathbf{in}}}
\\
    \ensuremath{(\mathbb{Z},\mathbb{Z}}^*\ensuremath{)}
&
    \ensuremath{\mathrm{1}+(\mathbb{Z},((\mathbb{Z},\mathbb{Z}}^*\ensuremath{),(\mathbb{Z},\mathbb{Z}}^*\ensuremath{)))}
           \ar[l]^-{\ensuremath{\Varid{f}\mathrel{=}\alt{\underline{\mathrm{0},[\mskip1.5mu \mskip1.5mu]}}{\Varid{hI}}}}
}
\end{eqnarray*}

Esta função é composta por um anamorfismo (\ensuremath{\Varid{anaBTree}\;\Varid{qsep}}) e por um catamorfismo (\ensuremath{\Varid{cataBTree}\;\alt{\underline{\mathrm{0},[\mskip1.5mu \mskip1.5mu]}}{\Varid{hI}}}).

\textbf{1. Anamorfismo}

A função \ensuremath{\Varid{qsep}} é responsável por dividir a lista de alturas do histograma e construir recursivamente a árvore binária.
Assim, caso a lista esteja vazia, retorna \ensuremath{i_1\;()}.

Caso contrário, o primeiro elemento da lista é escolhido como pivô e os elementos restantes são divididos em dois subconjuntos: \ensuremath{\Varid{s}} contém os elementos menores que o pivô e \ensuremath{\Varid{l}} contém os elementos maiores ou iguais ao pivô.
Esta divisão é realizada pela função \ensuremath{\Varid{part}} que percorre a lista e verifica, para cada elemento, se este satisfaz o predicado \ensuremath{\Varid{p}}, no caso da função \ensuremath{\Varid{qsep}}, se é menor que o pivô.

Então, o resultado da função \ensuremath{\Varid{qsep}} é uma árvore binária onde cada nodo contém um pivô e as suas subárvores representam os valores menores e maiores, respetivamente.

\textbf{2. Catamorfismo}

O catamorfismo \ensuremath{\Varid{cataBTree}\;\alt{\underline{\mathrm{0},[\mskip1.5mu \mskip1.5mu]}}{\Varid{hI}}} verifica se o nodo é vazio e retorna \ensuremath{(\mathrm{0},[\mskip1.5mu \mskip1.5mu])}.
Caso contrário, aplica a função \ensuremath{\Varid{hI}} que calcula o h-index e os contribuidores para o nodo atual.

A função \ensuremath{\Varid{hI}} segue os seguintes passos:

\textbf{2.1. Combinação dos valores das subárvores:} junta os valores das subárvores esquerda e direita numa lista, adicionando o valor do nodo atual.

\textbf{2.2. Cálculo do h-index:} cada elemento da lista é emparelhado com a sua posição usando \ensuremath{\Varid{zip}\;[\mskip1.5mu \mathrm{1}\mathinner{\ldotp\ldotp}\mskip1.5mu]\;\Varid{list}}, a função \ensuremath{\Varid{myfoldr}} percorre esses pares para calcular o maior índice \ensuremath{\Varid{k}} tal que o valor associado seja maior ou igual a \ensuremath{\Varid{k}}.
Ou seja, a lista \ensuremath{\Varid{list}} é transformada em pares \ensuremath{(\Varid{k},\Varid{height})}, onde \ensuremath{\Varid{k}} representa a posição e \ensuremath{\Varid{height}} é o alor da altura correspondente. A função \ensuremath{\Varid{process}} verifica:
\begin{itemize}
    \item Se \ensuremath{\Varid{height}\geq \Varid{k}}, então o h-index é atualizado para o máximo entre o valor atual e \ensuremath{\Varid{k}}.
    \item Caso contrário, o h-index mantém-se inalterado.
\end{itemize}
A função \ensuremath{\Varid{process}}:
\begin{itemize}
    \item verifica se a altura é maior ou igual ao índice: \ensuremath{\uncurry{(\geq )}\comp \Varid{swap}\comp \p1};
    \item se a condição for satisfeita, atualiza o h-index: \ensuremath{\uncurry{\Varid{max}}\comp \Varid{swap}\comp (\p1\times\Varid{id})};
    \item caso contrário, mantém o valor atual: \ensuremath{\p2}.
\end{itemize}

\textbf{2.3. Identificação dos contribuidores:} a lista é filtrada para conter apenas os valores maiores ou iguais ao h-index (\ensuremath{\Varid{filter}\;(\geq \Varid{hIndex})\;\Varid{list}}).

Segue a implementação da função \ensuremath{\Varid{hindex}}:
\begin{hscode}\SaveRestoreHook
\column{B}{@{}>{\hspre}l<{\hspost}@{}}%
\column{5}{@{}>{\hspre}l<{\hspost}@{}}%
\column{9}{@{}>{\hspre}l<{\hspost}@{}}%
\column{E}{@{}>{\hspre}l<{\hspost}@{}}%
\>[B]{}\Varid{hindex}\mathrel{=}\Varid{hyloBTree}\;\alt{\underline{\mathrm{0},[\mskip1.5mu \mskip1.5mu]}}{\Varid{hI}}\;\Varid{qsep}{}\<[E]%
\\[\blanklineskip]%
\>[B]{}\Varid{hI}\mathbin{::}(\Conid{Int},((\Conid{Int},[\mskip1.5mu \Conid{Int}\mskip1.5mu]),(\Conid{Int},[\mskip1.5mu \Conid{Int}\mskip1.5mu])))\to (\Conid{Int},[\mskip1.5mu \Conid{Int}\mskip1.5mu]){}\<[E]%
\\
\>[B]{}\Varid{hI}\;(\Varid{n},((\anonymous ,\Varid{ll}),(\anonymous ,\Varid{lr})))\mathrel{=}(\Varid{hIndex},\Varid{contributors}){}\<[E]%
\\
\>[B]{}\hsindent{5}{}\<[5]%
\>[5]{}\mathbf{where}{}\<[E]%
\\
\>[5]{}\hsindent{4}{}\<[9]%
\>[9]{}\Varid{list}\mathrel{=}\Varid{lr}\mathbin{+\!\!+}[\mskip1.5mu \Varid{n}\mskip1.5mu]\mathbin{+\!\!+}\Varid{ll}{}\<[E]%
\\
\>[5]{}\hsindent{4}{}\<[9]%
\>[9]{}\Varid{hIndex}\mathrel{=}\Varid{myfoldr}\;\overline{\Varid{process}}\;\mathrm{0}\;(\Varid{zip}\;[\mskip1.5mu \mathrm{1}\mathinner{\ldotp\ldotp}\mskip1.5mu]\;\Varid{list}){}\<[E]%
\\
\>[5]{}\hsindent{4}{}\<[9]%
\>[9]{}\Varid{process}\mathbin{::}(\Conid{Ord}\;\Varid{a})\Rightarrow ((\Varid{a},\Varid{a}),\Varid{a})\to \Varid{a}{}\<[E]%
\\
\>[5]{}\hsindent{4}{}\<[9]%
\>[9]{}\Varid{process}\mathrel{=}\Varid{cond}\;(\uncurry{(\geq )}\comp \Varid{swap}\comp \p1)\;(\uncurry{\Varid{max}}\comp \Varid{swap}\comp (\p1\times\Varid{id}))\;\p2{}\<[E]%
\\
\>[5]{}\hsindent{4}{}\<[9]%
\>[9]{}\Varid{contributors}\mathrel{=}\Varid{filter}\;(\geq \Varid{hIndex})\;\Varid{list}{}\<[E]%
\ColumnHook
\end{hscode}\resethooks

\subsection*{Problema 2}

Primeira parte:

A função \ensuremath{\Varid{primes}} é responsável por criar a lista de fatores primos de um dado número. De modo que, esta função pode ser definida
como um anamorfismo de listas (\ensuremath{\Conid{List}}). Assim, o diagrama que representa a operação é o seguinte:

\begin{eqnarray*}
\xymatrix@C=2cm{
    \ensuremath{\mathbb{Z}}
           \ar[d]_-{\ensuremath{\anaList{\Varid{g}}}}
            \ar[r]^-{\ensuremath{\Varid{g}}} 
&
    \ensuremath{\mathrm{1}\mathbin{+}\mathbb{Z}} \times \ensuremath{\mathbb{Z}}
           \ar[d]^{\ensuremath{\Varid{id}\mathbin{+}(\Varid{id}} \times \ensuremath{\anaList{\Varid{g}}})}
\\
     \ensuremath{\mathbb{Z}}^*
            \ar@/_/[r]_-{\ensuremath{\Varid{out}}_\ensuremath{\Conid{List}}} 
&
     \ensuremath{\mathrm{1}\mathbin{+}\mathbb{Z}} \times \ensuremath{\mathbb{Z}}^*
           \ar@/_/[l]_-{\ensuremath{\mathbf{in}}_\ensuremath{\Conid{List}}}
}
\end{eqnarray*}

A implementação baseia-se em decompor o número repetidamente no seu menor fator primo, este processo repete-se até que o quociente resultante seja 1.

O processo pode ser representado graficamente como se segue para o número 455:

\begin{eqnarray*}
\xymatrix@C=2cm{
    \ensuremath{\mathrm{455}}
        \ar[d]
\\
    \ensuremath{(\mathrm{5},\mathrm{91})}
        \ar[d]
\\
    \ensuremath{(\mathrm{7},\mathrm{13})}
        \ar[d]
\\
    \ensuremath{(\mathrm{13},\mathrm{1})}
        \ar[d]
\\
    \ensuremath{[\mskip1.5mu \mskip1.5mu]}
}
\end{eqnarray*}

Assim, \ensuremath{\Varid{primes}\;\mathrm{455}\mathrel{=}[\mskip1.5mu \mathrm{5},\mathrm{7},\mathrm{13}\mskip1.5mu]}.

A definição de \ensuremath{\Varid{primes}} como \ensuremath{\anaList{\Varid{g}}} tira partido de que um anamorfismo constrói uma estrutura recursiva ao aplicar sucessivamente o gene \ensuremath{\Varid{g}} a um valor inicial.
O gene \ensuremath{\Varid{g}} determina como cada passo da construção ocorre, neste caso \ensuremath{\Varid{g}} divide o número \ensuremath{\Varid{n}} no seu menor fator primo (calculado pela função \ensuremath{\Varid{smallestPrimeFactor}}) e no quociente resultante após a divisão.
O processo termina quando \ensuremath{\Varid{n}\mathrel{=}\mathrm{1}}, porque não existem mais fatores primos para serem determinados.

A função \ensuremath{\Varid{smallestPrimeFactor}} é responsável por determinar o menor fator primo de um número \ensuremath{\Varid{n}}, e é definida como um catamorfismo de naturais (\ensuremath{\Varid{catNat}}).
Esta função aplica sucessivamente a lógica de "testar se um divisor \ensuremath{\Varid{d}} divide \ensuremath{\Varid{n}}" para valores \ensuremath{\Varid{d}} crescentes, assim inicia com o menor número primo (\ensuremath{\mathrm{2}}).

O ciclo-for contém uma estrutura recursiva que verifica duas condições:

1. \textbf{Teste de primalidade:} Se \begin{math} d^2 > n\end{math}: Nesse caso, \ensuremath{\Varid{n}} é primo e o seu menor fator primo é ele mesmo (o processo termina).

2. \textbf{Encontrar o menor fator primo:} Se \begin{math}n\mod d = 0\end{math}: Nesse caso, \ensuremath{\Varid{d}} é o menor fator primo de \ensuremath{\Varid{n}}.

\textbf{Caso contrário:} Incrementámos \ensuremath{\Varid{d}} e continuámos o processo.
\paragraph{}
\textbf{Fundamentação matemática:}
A implementação baseia-se no Teorema Fundamental da Aritmética, que garante que todo o número inteiro positivo maior que 1
pode ser decomposto de forma única como um produto de fatores primos.
O processo descrito no gene \ensuremath{\Varid{g}} utiliza esta propriedade para decompor iterativamente o \ensuremath{\Varid{n}} nos seus fatores primos,
onde a divisibilidade é verificada e avançamos na procura do menor fator primo.

\begin{hscode}\SaveRestoreHook
\column{B}{@{}>{\hspre}l<{\hspost}@{}}%
\column{38}{@{}>{\hspre}l<{\hspost}@{}}%
\column{E}{@{}>{\hspre}l<{\hspost}@{}}%
\>[B]{}\Varid{smallestPrimeFactor}\;\Varid{x}\mathrel{=}\for{\lambda \Varid{n}\to \Varid{cond}\;(\uncurry{(\mathbin{>})}\comp ((\mathbin{\uparrow}\mathrm{2})\times\Varid{id}))\;\p2\;{}\<[E]%
\\
\>[B]{}\hsindent{38}{}\<[38]%
\>[38]{}(\Varid{cond}\;((\equiv \mathrm{0})\comp \uncurry{\Varid{mod}}\comp \Varid{swap})\;\p1\;(\succ \comp \p1))\;(\Varid{n},\Varid{x})}\ {\mathrm{2}}\;\Varid{x}{}\<[E]%
\\[\blanklineskip]%
\>[B]{}\Varid{g}\;\mathrm{1}\mathrel{=}i_1\;(){}\<[E]%
\\
\>[B]{}\Varid{g}\;\Varid{n}\mathrel{=}i_2\;(\Varid{smallestPrimeFactor}\;\Varid{n},\Varid{n}\div \Varid{smallestPrimeFactor}\;\Varid{n}){}\<[E]%
\\[\blanklineskip]%
\>[B]{}\Varid{primes}\mathrel{=}\anaList{\Varid{g}}{}\<[E]%
\ColumnHook
\end{hscode}\resethooks

Segunda parte:

A função \ensuremath{\Varid{prime\char95 tree}} é responsável por criar a árvore dos primos de uma lista de inteiros, como se encontra ilustrado no enunciado.
De modo que, esta função pode ser definida da seguinte forma:

\begin{hscode}\SaveRestoreHook
\column{B}{@{}>{\hspre}l<{\hspost}@{}}%
\column{E}{@{}>{\hspre}l<{\hspost}@{}}%
\>[B]{}\Varid{prime\char95 tree}\mathrel{=}\Conid{Term}\;\mathrm{1}\comp \Varid{untar}\comp \map \;(\lambda \Varid{n}\to (\Varid{primes}\;\Varid{n},\Varid{n})){}\<[E]%
\ColumnHook
\end{hscode}\resethooks

Inicialmente, adotámos uma abordagem extensiva para resolver o problema, com a definição de um hilomorfismo e todas as operações necessárias para construir a árvore.
No entanto, durante este processo, reparámos na função \ensuremath{\Varid{untar}} da biblioteca \ensuremath{\Varid{\Conid{Exp}.hs}}, que efetua a operação necessária para transformar uma lista de pares numa estrutura do tipo \ensuremath{[\mskip1.5mu \Conid{Exp}\;\Varid{v}\;\Varid{o}\mskip1.5mu]}.
Após compreendermos o comportamento e a definição da função \ensuremath{\Varid{untar}}, percebemos que era possível utilizá-la na construção da função \ensuremath{\Varid{prime\char95 tree}}, o que simplificou a implementação.

Explicação da função \ensuremath{\Varid{prime\char95 tree}}:

1. A função \ensuremath{\Varid{primes}} é aplicada a cada elemento da lista de inteiros e com o uso da expressão \ensuremath{\map \;(\lambda \Varid{n}\to (\Varid{primes}\;\Varid{n},\Varid{n}))}, obtemos uma lista de pares, onde o primeiro elemento é a lista de fatores primos de um número e o segundo elemento é o próprio número.
Assim, no final da execução desta expressão, obtemos uma lista de pares do tipo \ensuremath{[\mskip1.5mu ([\mskip1.5mu \mathbb{Z}\mskip1.5mu],\mathbb{Z})\mskip1.5mu]}.

2. Neste contexto, a função \ensuremath{\Varid{untar}} converte os fatores primos de um número e o próprio número numa representação de árvore onde os nodos intermediários são os fatores e as folhas são os números originais, \ensuremath{[\mskip1.5mu \Conid{Exp}\;\mathbb{Z}\;\mathbb{Z}\mskip1.5mu]}.
Esta conversão é realizada em três partes principais: a coalgebra, a base e a álgebra.

2.1. A coalgebra, representada pela função \ensuremath{\Varid{c}}, é responsável por decompor os dados, ou seja, separa os pares da lista inicial - \ensuremath{[\mskip1.5mu ([\mskip1.5mu \mathbb{Z}\mskip1.5mu],\mathbb{Z})\mskip1.5mu]} - e transforma cada elemento para o formato \ensuremath{\mathbb{Z}+(\mathbb{Z},[\mskip1.5mu ([\mskip1.5mu \mathbb{Z}\mskip1.5mu],\mathbb{Z})\mskip1.5mu])}.

2.1.1. O \ensuremath{\map \;((\p2\mathbin{+}\Varid{assocr})\comp \Varid{distl}\comp (\Varid{outList}\times\Varid{id}))} é aplicado a cada par da lista, onde:
    \begin{itemize}
        \item \ensuremath{\Varid{outList}\times\Varid{id}} transforma a lista de fatores primos num tipo \ensuremath{\cdot +\cdot } e retorna o número original. Permitindo tratar em separado os fatores primos e os números. 
        \item \ensuremath{\Varid{distl}} distribui os elementos \ensuremath{(\Varid{a},\Varid{b})+\cdot } para o tipo \ensuremath{(\Varid{a}+\Varid{b},\Varid{b})}, separa os dados para facilitar o processamento posterior.
        \item \ensuremath{\p2\mathbin{+}\Varid{assocr}} reorganiza os pares para agrupar corretamente os fatores primos associados a um número.
    \end{itemize}

2.1.2. \ensuremath{\Varid{sep}} percorre a lista de elementos \ensuremath{\cdot +\cdot }, e separa os elementos \ensuremath{i_1} para o primeiro grupo e os \ensuremath{i_2} para o segundo grupo.

2.1.3. \ensuremath{\Varid{id}\times\Varid{collect}} aplica a função \ensuremath{\Varid{id}} ao primeiro valor do tuplo e \ensuremath{\Varid{collect}} ao segundo, de modo que a função \ensuremath{\Varid{collect}} é responsável por agrupar os fatores primos em listas separadas para cada número.
Então, os números que partilham o mesmo fator primo são agrupados juntos.

2.1.4. \ensuremath{\Varid{join}} junta os valores numa lista única, recriando a estrutura necessária para representare os dados.

2.2. Após a coalgebra, avançamos para a base, esta aplica recursivamente a função \ensuremath{\Varid{untar}} a cada sublista e cria subárvores para cada conjunto de fatores. O tipo da função \ensuremath{\Varid{base}} é definido como:

\ensuremath{\Varid{base}\mathbin{::}(\mathbb{Z}\to \mathbb{Z})\to (\mathbb{Z}\to \mathbb{Z})\to ([\mskip1.5mu ([\mskip1.5mu \mathbb{Z}\mskip1.5mu],\mathbb{Z})\mskip1.5mu]\to [\mskip1.5mu \Conid{Exp}\;\mathbb{Z}\;\mathbb{Z}\mskip1.5mu])\to [\mskip1.5mu \mathbb{Z}+(\mathbb{Z},[\mskip1.5mu ([\mskip1.5mu \mathbb{Z}\mskip1.5mu],\mathbb{Z})\mskip1.5mu])\mskip1.5mu]\to [\mskip1.5mu \mathbb{Z}+(\mathbb{Z},[\mskip1.5mu \Conid{Exp}\;\mathbb{Z}\;\mathbb{Z}\mskip1.5mu])\mskip1.5mu]}.

2.3. Para finalizar, a álgebra \ensuremath{\Varid{a}} organiza os dados processados na estrutura final, a operação \ensuremath{\Varid{sort}} organiza os nodos, enquanto que o \ensuremath{\map \;\Varid{inExp}} converte os elementos numa estrutura do tipo \ensuremath{\Conid{Exp}\;\mathbb{Z}\;\mathbb{Z}}.
O seu tipo, neste contexto, é definido como: \ensuremath{\Varid{a}\mathbin{::}[\mskip1.5mu \mathbb{Z}+(\mathbb{Z},[\mskip1.5mu \Conid{Exp}\;\mathbb{Z}\;\mathbb{Z}\mskip1.5mu])\mskip1.5mu]\to [\mskip1.5mu \Conid{Exp}\;\mathbb{Z}\;\mathbb{Z}\mskip1.5mu]}.

3. Por fim, a função \ensuremath{\Conid{Term}\;\mathrm{1}} é aplicada para adicionar a raíz da árvore com o valor 1, isto conecta todas as subárvores criadas pela função \ensuremath{\Varid{untar}}, construindo uma única árvore que representa a decomposição de todos os números da lista.

\subsection*{Problema 3}

\begin{hscode}\SaveRestoreHook
\column{B}{@{}>{\hspre}l<{\hspost}@{}}%
\column{5}{@{}>{\hspre}l<{\hspost}@{}}%
\column{9}{@{}>{\hspre}l<{\hspost}@{}}%
\column{21}{@{}>{\hspre}l<{\hspost}@{}}%
\column{37}{@{}>{\hspre}l<{\hspost}@{}}%
\column{E}{@{}>{\hspre}l<{\hspost}@{}}%
\>[B]{}\Varid{convolve}\mathbin{::}\Conid{Num}\;\Varid{a}\Rightarrow [\mskip1.5mu \Varid{a}\mskip1.5mu]\to [\mskip1.5mu \Varid{a}\mskip1.5mu]\to [\mskip1.5mu \Varid{a}\mskip1.5mu]{}\<[E]%
\\
\>[B]{}\Varid{convolve}\;\Varid{l1}\;l_2 \mathrel{=}\anaList{\Varid{anaGene}\;\Varid{l1}\;l_2 }\;\mathrm{0}{}\<[E]%
\\[\blanklineskip]%
\>[B]{}\Varid{anaGene}\mathbin{::}\Conid{Num}\;\Varid{a}\Rightarrow [\mskip1.5mu \Varid{a}\mskip1.5mu]\to [\mskip1.5mu \Varid{a}\mskip1.5mu]\to \Conid{Int}\to ()+(\Varid{a},\Conid{Int}){}\<[E]%
\\
\>[B]{}\Varid{anaGene}\;\Varid{l1}\;l_2 \mathrel{=}\Varid{cond}\;(\geq \Varid{m}\mathbin{+}\Varid{n}\mathbin{-}\mathrm{1})\;(\underline{\cdot }\mathbin{\$}i_1\;())\;{}\<[E]%
\\
\>[B]{}\hsindent{21}{}\<[21]%
\>[21]{}(\lambda \Varid{i}\to i_2\;(\Varid{sum}{}\<[37]%
\>[37]{}\mathbin{\$}\Varid{zipWith}\;(\mathbin{*})\;\Varid{l1}\;(\map \;(\lambda \Varid{j}\to \Varid{access}\;(l_2 ,(\Varid{i},\Varid{j})))\;[\mskip1.5mu \mathrm{0}\mathinner{\ldotp\ldotp}(\Varid{m}\mathbin{-}\mathrm{1})\mskip1.5mu]),\Varid{i}\mathbin{+}\mathrm{1})){}\<[E]%
\\
\>[B]{}\hsindent{5}{}\<[5]%
\>[5]{}\mathbf{where}{}\<[E]%
\\
\>[5]{}\hsindent{4}{}\<[9]%
\>[9]{}\Varid{m}\mathrel{=}\length \;\Varid{l1};\Varid{n}\mathrel{=}\length \;l_2 {}\<[E]%
\\
\>[5]{}\hsindent{4}{}\<[9]%
\>[9]{}\Varid{access}\mathrel{=}\Varid{cond}\;((\mathrel{\vee})\mathbin{\mathopen{\langle}\$\mathclose{\rangle}}\Varid{cond1}\mathbin{<*>}\Varid{cond2})\;\underline{\mathrm{0}}\;(\uncurry{(\mathbin{!!})}\comp (\Varid{id}\times\uncurry{(\mathbin{-})})){}\<[E]%
\\
\>[5]{}\hsindent{4}{}\<[9]%
\>[9]{}\Varid{cond1}\mathrel{=}\uncurry{(\mathbin{>})}\comp (\underline{\mathrm{0}}\times\uncurry{(\mathbin{-})}){}\<[E]%
\\
\>[5]{}\hsindent{4}{}\<[9]%
\>[9]{}\Varid{cond2}\mathrel{=}\uncurry{(\leq )}\comp (\length \times\uncurry{(\mathbin{-})}){}\<[E]%
\ColumnHook
\end{hscode}\resethooks

\subsection*{Problema 4}

1. Nesta pergunta, pretende-se definir por completo, a biblioteca \ensuremath{\Conid{Expr}}, à semelhança das outras bibliotecas de estruturas fornecidas.

Definição do tipo de \ensuremath{\Conid{Expr}}:

Para o cálulo de \ensuremath{\Varid{outExpr}}, partimos da afirmação \ensuremath{\Varid{outExpr}\comp \Varid{inExpr}\mathrel{=}\Varid{id}},

\begin{eqnarray*}
\start
\ensuremath{\Varid{outExpr}\comp \Varid{inExpr}\mathrel{=}\Varid{id}}
\just\equiv{def inExpr}
\ensuremath{\Varid{outExpr}\comp \alt{\Conid{V}}{\alt{\Conid{N}}{\uncurry{\Conid{T}}}}\mathrel{=}\Varid{id}}
\just\equiv{ fusão + }
\ensuremath{\alt{\Varid{outExpr}\comp \Conid{V}}{\alt{\Varid{outExpr}\comp \Conid{N}}{\Varid{outExpr}\comp \uncurry{\Conid{T}}}}\mathrel{=}\Varid{id}}
\just\equiv{ universal +, natural id }
\ensuremath{\begin{lcbr}\Varid{outExpr}\comp \Conid{V}\mathrel{=}i_1\\\alt{\Varid{outExpr}\comp \Conid{N}}{\Varid{outExpr}\comp \uncurry{\Conid{T}}}\mathrel{=}i_2\end{lcbr}}
\just\equiv{ universal + }
\ensuremath{\begin{lcbr}\Varid{outExpr}\comp \Conid{V}\mathrel{=}i_1\\\Varid{outExpr}\comp \Conid{N}\mathrel{=}i_2\comp i_1\\\Varid{outExpr}\comp \uncurry{\Conid{T}}\mathrel{=}i_2\comp i_2\end{lcbr}}
\just\equiv{ igualdade extensional, def-comp, uncurry }
\ensuremath{\begin{lcbr}\Varid{outExpr}\;(\Conid{V}\;\Varid{n})\mathrel{=}i_1\;\Varid{n}\\\Varid{outExpr}\;(\Conid{N}\;\Varid{n})\mathrel{=}(i_2\comp i_1)\;\Varid{n}\\\Varid{outExpr}\;(\Conid{T}\;\Varid{op}\;\Varid{exprs})\mathrel{=}(i_2\comp i_2)\;(\Varid{op},\Varid{exprs})\end{lcbr}}
\end{eqnarray*}

Ficando assim, em Haskell, com:

\begin{hscode}\SaveRestoreHook
\column{B}{@{}>{\hspre}l<{\hspost}@{}}%
\column{E}{@{}>{\hspre}l<{\hspost}@{}}%
\>[B]{}\Varid{outExpr}\mathbin{::}\Conid{Expr}\;\Varid{b}\;\Varid{a}\to \Varid{a}+(\Varid{b}+(\Conid{Op},[\mskip1.5mu \Conid{Expr}\;\Varid{b}\;\Varid{a}\mskip1.5mu])){}\<[E]%
\\
\>[B]{}\Varid{outExpr}\;(\Conid{V}\;\Varid{n})\mathrel{=}i_1\;\Varid{n}{}\<[E]%
\\
\>[B]{}\Varid{outExpr}\;(\Conid{N}\;\Varid{n})\mathrel{=}(i_2\comp i_1)\;\Varid{n}{}\<[E]%
\\
\>[B]{}\Varid{outExpr}\;(\Conid{T}\;\Varid{op}\;\Varid{exprs})\mathrel{=}(i_2\comp i_2)\;(\Varid{op},\Varid{exprs}){}\<[E]%
\ColumnHook
\end{hscode}\resethooks

Cálculo do functor de \ensuremath{\Conid{Expr}}:

Sabendo que \ensuremath{\Conid{F}\;\Varid{f}\mathrel{=}\Conid{B}\;(\Varid{id},\Varid{f})}, temos que:

\begin{eqnarray*}
\start
\ensuremath{\Conid{F}\;\Varid{f}\mathrel{=}\Conid{B}\;(\Varid{id},\Varid{id},\Varid{f})}
\just\equiv{ def B }
\ensuremath{\Conid{F}\;\Varid{f}\mathrel{=}\Varid{id}\mathbin{+}(\Varid{id}\mathbin{+}\Varid{id}\times\map \;\Varid{f})}
\end{eqnarray*}

Definindo, então, \ensuremath{\Varid{recExpr}} como:

\begin{hscode}\SaveRestoreHook
\column{B}{@{}>{\hspre}l<{\hspost}@{}}%
\column{E}{@{}>{\hspre}l<{\hspost}@{}}%
\>[B]{}\Varid{recExpr}\mathbin{::}(\Varid{a}\to \Varid{b1})\to \Varid{b2}+(\Varid{b3}+(\Varid{b4},[\mskip1.5mu \Varid{a}\mskip1.5mu]))\to \Varid{b2}+(\Varid{b3}+(\Varid{b4},[\mskip1.5mu \Varid{b1}\mskip1.5mu])){}\<[E]%
\\
\>[B]{}\Varid{recExpr}\mathrel{=}\Varid{baseExpr}\;\Varid{id}\;\Varid{id}{}\<[E]%
\ColumnHook
\end{hscode}\resethooks

Definição da triologia ana-cata-hylo:

Começando pelo catamorfismo de \ensuremath{\Conid{Expr}}, temos:

\begin{eqnarray*}
\start
\just\equiv{ cancelamento-cata }
\ensuremath{\llparenthesis\, \Varid{g}\,\rrparenthesis\comp \mathsf{in}\mathrel{=}\Varid{g}\comp \fun F \;\llparenthesis\, \Varid{g}\,\rrparenthesis}
\just\equiv{ shunt-left }
\ensuremath{\llparenthesis\, \Varid{g}\,\rrparenthesis\mathrel{=}\Varid{g}\comp \fun F \;\llparenthesis\, \Varid{g}\,\rrparenthesis\comp \mathsf{out}}
\end{eqnarray*}

Representado pelo seguinte diagrama:

\begin{eqnarray*}
\xymatrix@C=2cm{
    \ensuremath{\Conid{Expr}\;\Conid{C}\;\Conid{A}}
           \ar[d]_-{\ensuremath{\cataNat{\Varid{g}}}}
           \ar@/^-1pc/[r]_-{\ensuremath{\Varid{out}}}
&
    \ensuremath{\Conid{A}\mathbin{+}(\Conid{C}\mathbin{+}(\Conid{Op}\times(\Conid{Expr}\;\Conid{C}\;\Conid{A})}^*))
           \ar[d]^{\ensuremath{\Varid{id}\mathbin{+}(\Varid{id}\mathbin{+}(\Varid{id}\times\map \;\cataNat{\Varid{g}}))}}
           \ar@/^-1pc/[l]_-{\ensuremath{\mathsf{in}}}
\\
    \ensuremath{\Conid{Expr}\;\Conid{C}\;\Conid{B}}
&
    \ensuremath{\Conid{A}\mathbin{+}(\Conid{C}\mathbin{+}(\Conid{Op}\times(\Conid{Expr}\;\Conid{C}\;\Conid{B})}^*))
           \ar[l]^-{\ensuremath{\Varid{g}}}
}
\end{eqnarray*}

Utilizando as funções previamente definidas, temos então:

\begin{hscode}\SaveRestoreHook
\column{B}{@{}>{\hspre}l<{\hspost}@{}}%
\column{E}{@{}>{\hspre}l<{\hspost}@{}}%
\>[B]{}\Varid{cataExpr}\;\Varid{g}\mathrel{=}\Varid{g}\comp \Varid{recExpr}\;(\Varid{cataExpr}\;\Varid{g})\comp \Varid{outExpr}{}\<[E]%
\ColumnHook
\end{hscode}\resethooks

Para o anamorfismo de \ensuremath{\Conid{Expr}}, temos:

\begin{eqnarray*}
\start
\just\equiv{ cancelamento-ana }
\ensuremath{\mathsf{out}\comp \ana{\Varid{g}}\mathrel{=}\fun F \;\ana{\Varid{g}}\comp \Varid{g}}
\just\equiv{ shunt-right }
\ensuremath{\ana{\Varid{g}}\mathrel{=}\mathsf{in}\comp \fun F \;\ana{\Varid{g}}\comp \Varid{g}}
\end{eqnarray*}

Representado pelo seguinte diagrama:

\begin{eqnarray*}
\xymatrix@C=2cm{
    \ensuremath{\Conid{Expr}\;\Conid{C}\;\Conid{A}}
           \ar@/^-1pc/[r]_-{\ensuremath{\Varid{out}}}
&
    \ensuremath{\Conid{A}\mathbin{+}(\Conid{C}\mathbin{+}(\Conid{Op}\times(\Conid{Expr}\;\Conid{C}\;\Conid{A})}^*))
           \ar@/^-1pc/[l]_-{\ensuremath{\mathsf{in}}}
\\
    \ensuremath{\Conid{Expr}\;\Conid{C}\;\Conid{D}}
            \ar[u]^-{\ensuremath{\ana{\Varid{g}}}}
            \ar[r]_-{\ensuremath{\Varid{g}}}
&
    \ensuremath{\Conid{A}\mathbin{+}(\Conid{C}\mathbin{+}(\Conid{Op}\times(\Conid{Expr}\;\Conid{C}\;\Conid{D})}^*))
            \ar[u]_-{\ensuremath{\Varid{id}\mathbin{+}(\Varid{id}\mathbin{+}(\Varid{id}\times\map \;\ana{\Varid{g}}))}}
}
\end{eqnarray*}

Utilizando as funções previamente definidas, temos então:

\begin{hscode}\SaveRestoreHook
\column{B}{@{}>{\hspre}l<{\hspost}@{}}%
\column{E}{@{}>{\hspre}l<{\hspost}@{}}%
\>[B]{}\Varid{anaExpr}\;\Varid{g}\mathrel{=}\Varid{inExpr}\comp \Varid{recExpr}\;(\Varid{anaExpr}\;\Varid{g})\comp \Varid{g}{}\<[E]%
\ColumnHook
\end{hscode}\resethooks

Sendo o hilomorfismo, a composição do catamorfismo e do anamorfismo, representado pelo seguinte diagrama:

\begin{eqnarray*}
\xymatrix@C=2cm{
    \ensuremath{\Conid{Expr}\;\Conid{C}\;\Conid{D}}
           \ar[d]_-{\ensuremath{\ana{\Varid{g}}}}
           \ar[r]_-{\ensuremath{\Varid{g}}}
&
    \ensuremath{\Conid{A}\mathbin{+}(\Conid{C}\mathbin{+}(\Conid{Op}\times(\Conid{Expr}\;\Conid{C}\;\Conid{D})}^*))
           \ar[d]^{\ensuremath{\Varid{id}\mathbin{+}(\Varid{id}\mathbin{+}(\Varid{id}\times\map \;\ana{\Varid{g}}))}}
\\
    \ensuremath{\Conid{Expr}\;\Conid{C}\;\Conid{A}}
           \ar[d]_-{\ensuremath{\cataNat{\Varid{h}}}}
           \ar@/^-1pc/[r]_-{\ensuremath{\Varid{out}}}
&
    \ensuremath{\Conid{A}\mathbin{+}(\Conid{C}\mathbin{+}(\Conid{Op}\times(\Conid{Expr}\;\Conid{C}\;\Conid{A})}^*))
           \ar[d]^{\ensuremath{\Varid{id}\mathbin{+}(\Varid{id}\mathbin{+}(\Varid{id}\times\map \;\cataNat{\Varid{h}}))}}
           \ar@/^-1pc/[l]_-{\ensuremath{\mathsf{in}}}
\\
    \ensuremath{\Conid{Expr}\;\Conid{C}\;\Conid{B}}
&
    \ensuremath{\Conid{A}\mathbin{+}(\Conid{C}\mathbin{+}(\Conid{Op}\times(\Conid{Expr}\;\Conid{C}\;\Conid{B})}^*))
           \ar[l]^-{\ensuremath{\Varid{h}}}
}
\end{eqnarray*}

ou seja,

\begin{eqnarray*}
\start
\ensuremath{\llbracket\, \Varid{h},\,\Varid{g}\,\rrbracket\mathrel{=}\llparenthesis\, \Varid{h}\,\rrparenthesis\comp \ana{\Varid{g}}}
\end{eqnarray*}

Este é definido em Haskell usando as funções \ensuremath{\Varid{cataExpr}} e \ensuremath{\Varid{anaExpr}} previamente definidas:

\begin{hscode}\SaveRestoreHook
\column{B}{@{}>{\hspre}l<{\hspost}@{}}%
\column{E}{@{}>{\hspre}l<{\hspost}@{}}%
\>[B]{}\Varid{hyloExpr}\;\Varid{h}\;\Varid{g}\mathrel{=}\Varid{cataExpr}\;\Varid{h}\comp \Varid{anaExpr}\;\Varid{g}{}\<[E]%
\ColumnHook
\end{hscode}\resethooks

\emph{Monad}:

Para declarar \ensuremath{\Conid{Expr}\;\Varid{b}} como instância da classe \ensuremath{\Conid{Monad}}, foram implementadas as intâncias de \ensuremath{\Conid{Functor}}, \ensuremath{\Conid{Applicative}} e \ensuremath{\Conid{Monad}} do tipo \ensuremath{\Conid{Expr}\;\Varid{b}}.
A abordagem utilizada foi guiada pelo exercício 4 da ficha 12, adaptando ao contexto específico de \ensuremath{\Conid{Expr}\;\Varid{b}}.

Começamos pelo \ensuremath{\Conid{Functor}}, onde a função \ensuremath{\mathsf{fmap}} aplica uma transformação às variáveis da expressão, mantendo as restantes estruturas (\ensuremath{\Conid{N}} e \ensuremath{\Conid{T}}) inalteradas.
Esta operação é realizada de forma recursiva usando o catamorfismo - \ensuremath{\Varid{cataExpr}} -, que reconstrói a expressão após aplicar a \ensuremath{\Varid{f}} às variáveis.
A composição com a função \ensuremath{\Varid{inExpr}} e a base do catamorfismo (\ensuremath{\Varid{baseExpr}}) assegura que a estrutura é processada corretamente.

\begin{hscode}\SaveRestoreHook
\column{B}{@{}>{\hspre}l<{\hspost}@{}}%
\column{6}{@{}>{\hspre}l<{\hspost}@{}}%
\column{E}{@{}>{\hspre}l<{\hspost}@{}}%
\>[B]{}\mathbf{instance}\;\Conid{Functor}\;(\Conid{Expr}\;\Varid{b}){}\<[E]%
\\
\>[B]{}\hsindent{6}{}\<[6]%
\>[6]{}\mathbf{where}\;\mathsf{fmap}\;\Varid{f}\mathrel{=}\Varid{cataExpr}\;(\Varid{inExpr}\comp \Varid{baseExpr}\;\Varid{f}\;\Varid{id}\;\Varid{id}){}\<[E]%
\ColumnHook
\end{hscode}\resethooks

De seguida, definimos a instância \ensuremath{\Conid{Applicative}}, onde a função \ensuremath{\Varid{pure}} cria uma expressão com uma variável (\ensuremath{\Conid{V}}) com o valor fornecido,
a função \ensuremath{\mathbin{<*>}} considera três casos:
\begin{itemize}
\item se a expressão é uma variável (\ensuremath{\Conid{V}\;\Varid{f}}), aplica \ensuremath{\mathsf{fmap}} para mapear função sobre a outra expressão;
\item se a expressão é um número (\ensuremath{\Conid{N}\;\Varid{b}}), mantém o número inalterado;
\item se a expressão é uma operação (\ensuremath{\Conid{T}\;\Varid{op}\;\Varid{fs}}), aplica a função a cada subexpressão da operação.
\end{itemize}

\begin{hscode}\SaveRestoreHook
\column{B}{@{}>{\hspre}l<{\hspost}@{}}%
\column{5}{@{}>{\hspre}l<{\hspost}@{}}%
\column{E}{@{}>{\hspre}l<{\hspost}@{}}%
\>[B]{}\mathbf{instance}\;\Conid{Applicative}\;(\Conid{Expr}\;\Varid{b})\;\mathbf{where}{}\<[E]%
\\
\>[B]{}\hsindent{5}{}\<[5]%
\>[5]{}\Varid{pure}\mathbin{::}\Varid{a}\to \Conid{Expr}\;\Varid{b}\;\Varid{a}{}\<[E]%
\\
\>[B]{}\hsindent{5}{}\<[5]%
\>[5]{}\Varid{pure}\mathrel{=}\Conid{V}{}\<[E]%
\\
\>[B]{}\hsindent{5}{}\<[5]%
\>[5]{}(\Conid{V}\;\Varid{f})\mathbin{<*>}\Varid{x}\mathrel{=}\mathsf{fmap}\;\Varid{f}\;\Varid{x}{}\<[E]%
\\
\>[B]{}\hsindent{5}{}\<[5]%
\>[5]{}(\Conid{N}\;\Varid{b})\mathbin{<*>}\anonymous \mathrel{=}\Conid{N}\;\Varid{b}{}\<[E]%
\\
\>[B]{}\hsindent{5}{}\<[5]%
\>[5]{}(\Conid{T}\;\Varid{op}\;\Varid{fs})\mathbin{<*>}\Varid{x}\mathrel{=}\Conid{T}\;\Varid{op}\;(\map \;(\mathbin{<*>}\Varid{x})\;\Varid{fs}){}\<[E]%
\ColumnHook
\end{hscode}\resethooks

Por fim, a instância \ensuremath{\Conid{Monad}} foi definida, a definição \ensuremath{\Varid{return}} é equivalente a \ensuremath{\Varid{pure}}, cria uma variável.
A operação \ensuremath{\bind } aplica a função \ensuremath{\Varid{g}} a cada variável da expressão, usando a função auxiliar \ensuremath{\Varid{muExpr}} para processar a expressão resultante.

\begin{hscode}\SaveRestoreHook
\column{B}{@{}>{\hspre}l<{\hspost}@{}}%
\column{5}{@{}>{\hspre}l<{\hspost}@{}}%
\column{9}{@{}>{\hspre}c<{\hspost}@{}}%
\column{9E}{@{}l@{}}%
\column{12}{@{}>{\hspre}l<{\hspost}@{}}%
\column{E}{@{}>{\hspre}l<{\hspost}@{}}%
\>[B]{}\mathbf{instance}\;\Conid{Monad}\;(\Conid{Expr}\;\Varid{b})\;\mathbf{where}{}\<[E]%
\\
\>[B]{}\hsindent{5}{}\<[5]%
\>[5]{}\Varid{return}\mathbin{::}\Varid{a}\to \Conid{Expr}\;\Varid{b}\;\Varid{a}{}\<[E]%
\\
\>[B]{}\hsindent{5}{}\<[5]%
\>[5]{}\Varid{return}\mathrel{=}\Varid{pure}{}\<[E]%
\\[\blanklineskip]%
\>[B]{}\hsindent{5}{}\<[5]%
\>[5]{}(\bind )\mathbin{::}\Conid{Expr}\;\Varid{b}\;\Varid{a}\to (\Varid{a}\to \Conid{Expr}\;\Varid{b}\;\Varid{b1})\to \Conid{Expr}\;\Varid{b}\;\Varid{b1}{}\<[E]%
\\
\>[B]{}\hsindent{5}{}\<[5]%
\>[5]{}\Varid{t}\bind \Varid{g}\mathrel{=}\Varid{muExpr}\;(\mathsf{fmap}\;\Varid{g}\;\Varid{t}){}\<[E]%
\\[\blanklineskip]%
\>[B]{}\Varid{muExpr}\mathbin{::}\Conid{Expr}\;\Varid{b}\;(\Conid{Expr}\;\Varid{b}\;\Varid{a})\to \Conid{Expr}\;\Varid{b}\;\Varid{a}{}\<[E]%
\\
\>[B]{}\Varid{muExpr}{}\<[9]%
\>[9]{}\mathrel{=}{}\<[9E]%
\>[12]{}\Varid{cataExpr}\;\alt{\Varid{id}}{\Varid{inExpr}\comp i_2}{}\<[E]%
\\[\blanklineskip]%
\>[B]{}\Varid{u}\mathbin{::}\Varid{a}\to \Conid{Expr}\;\Varid{b}\;\Varid{a}{}\<[E]%
\\
\>[B]{}\Varid{u}\mathrel{=}\Conid{V}{}\<[E]%
\ColumnHook
\end{hscode}\resethooks

Provemos que \ensuremath{\Conid{Expr}\;\Varid{b}} é uma instância de \ensuremath{\Conid{Monad}}:
\begin{itemize}
\item \ensuremath{\Varid{u}\mathrel{=}\Conid{V}} e \ensuremath{\Conid{V}\mathrel{=}\Varid{inExpr}\comp i_1}, logo \ensuremath{\Varid{u}\mathrel{=}\Varid{inExpr}\comp i_1};
\item \ensuremath{\Varid{muExpr}\mathrel{=}\Varid{cataExpr}\;\alt{\Varid{id}}{\Varid{inExpr}\comp i_2}}.
\end{itemize}

Provar a lei monádica Unidade (63):
\begin{eqnarray*}
\start
\ensuremath{\Varid{mu}\comp \Varid{u}\mathrel{=}\Varid{mu}\comp \Conid{T}\;\Varid{u}}
\just\equiv{ definição de mu; definição de u }
\ensuremath{\llparenthesis\, \alt{\Varid{id}}{\;\mathbf{in}\comp i_2}\,\rrparenthesis\comp \mathbf{in}\comp i_1\mathrel{=}\llparenthesis\, \alt{\Varid{id}}{\;\mathbf{in}\comp i_2}\,\rrparenthesis\comp \Conid{T}\;\Varid{u}}
\just\equiv{ absorção-cata }
\ensuremath{\llparenthesis\, \alt{\Varid{id}}{\;\mathbf{in}\comp i_2}\,\rrparenthesis\comp \mathbf{in}\comp i_1\mathrel{=}\llparenthesis\, \alt{\Varid{id}}{\;\mathbf{in}\comp i_2}\comp \Conid{B}\;(\Varid{u},\Varid{id})\,\rrparenthesis}
\just\equiv{ B(f,g) = f + G g, absorção-+, natural-id, functor-id-F }
\ensuremath{\llparenthesis\, \alt{\Varid{id}}{\;\mathbf{in}\comp i_2}\,\rrparenthesis\comp \mathbf{in}\comp i_1\mathrel{=}\llparenthesis\, \alt{\Varid{u}}{\;\mathbf{in}\comp i_2}\,\rrparenthesis}
\just\equiv{ definição de u, cancelamento-cata }
\ensuremath{\alt{\Varid{id}}{\;\mathbf{in}\comp i_2}\comp \Conid{F}\;\Varid{mu}\comp i_1\mathrel{=}\llparenthesis\, \alt{\;\mathbf{in}\comp i_1}{\;\mathbf{in}\comp i_2}\,\rrparenthesis}
\just\equiv{ fusão-+, reflexão-+, reflexão-cata, base-cata, B(id, mu) = id + G mu }
\ensuremath{\alt{\Varid{id}}{\;\mathbf{in}\comp i_2}\comp (\Varid{id}\mathbin{+}\Conid{G}\;\Varid{mu})\comp i_1\mathrel{=}\Varid{id}}
\just\equiv{ natural-i1, natural-id }
\ensuremath{\alt{\Varid{id}}{\;\mathbf{in}\comp i_2}\comp i_1\mathrel{=}\Varid{id}}
\just\equiv{ cancelamento-+ }
\ensuremath{\Varid{id}\mathrel{=}\Varid{id}}
\just\equiv{ P.R.I. }
\ensuremath{\Conid{True}}
\qed
\end{eqnarray*}

Provar a lei monádica Multiplicação (62):
\begin{eqnarray*}
\start
\ensuremath{\Varid{mu}\comp \Varid{mu}\mathrel{=}\Varid{mu}\comp \Conid{T}\;\Varid{mu}}
\just\equiv{ definição de mu }
\ensuremath{\Varid{mu}\comp \Varid{mu}\mathrel{=}\llparenthesis\, \alt{\Varid{id}}{\;\mathbf{in}\comp i_2}\,\rrparenthesis\comp \Conid{T}\;\llparenthesis\, \cdot \,\rrparenthesis\;\alt{\Varid{id}}{\;\mathbf{in}\comp i_2}}
\just\equiv{ absorção-cata }
\ensuremath{\Varid{mu}\comp \Varid{mu}\mathrel{=}\llparenthesis\, \alt{\Varid{id}}{\;\mathbf{in}\comp i_2}\comp (\llparenthesis\, \alt{\Varid{id}}{\;\mathbf{in}\comp i_2}\,\rrparenthesis\mathbin{+}\Conid{G}\;\Varid{id})\,\rrparenthesis}
\just\equiv{ Functor-id-F, natural-id, absorção-+ }
\ensuremath{\Varid{mu}\comp \Varid{mu}\mathrel{=}\llparenthesis\, \alt{\llparenthesis\, \alt{\Varid{id}}{\;\mathbf{in}\comp i_2}\,\rrparenthesis}{\;\mathbf{in}\comp i_2}\,\rrparenthesis}
\just\equiv{ definição de mu }
\ensuremath{\Varid{mu}\comp \llparenthesis\, \alt{\Varid{id}}{\;\mathbf{in}\comp i_2}\,\rrparenthesis\mathrel{=}\llparenthesis\, \alt{\llparenthesis\, \alt{\Varid{id}}{\;\mathbf{in}\comp i_2}\,\rrparenthesis}{\;\mathbf{in}\comp i_2}\,\rrparenthesis}
\just\Leftarrow{ fusão-cata }
\ensuremath{\Varid{mu}\comp \alt{\Varid{id}}{\;\mathbf{in}\comp i_2}\mathrel{=}\alt{\llparenthesis\, \alt{\Varid{id}}{\;\mathbf{in}\comp i_2}\,\rrparenthesis}{\;\mathbf{in}\comp i_2}\comp (\Varid{id}\mathbin{+}\Conid{G}\;\Varid{mu})}
\just\equiv{ fusão-+, absorção-+, natural-id, eq-+ }
\ensuremath{\alt{\Varid{id}}{\;\mathbf{in}\comp i_2}\comp i_1\mathrel{=}\Varid{id}}
\just\equiv{ cancelamento-+ }
\ensuremath{\begin{lcbr}\Varid{mu}\mathrel{=}\Varid{mu}\\\Varid{mu}\comp \mathbf{in}\comp i_2\mathrel{=}\mathbf{in}\comp i_2\comp \Conid{G}\;\Varid{mu}\end{lcbr}}
\just\equiv{ p \& True = True, definição de mu, cancelamento-cata, base-cata }
\ensuremath{\alt{\Varid{id}}{\;\mathbf{in}\comp i_2}\comp (\Varid{id}\mathbin{+}\Conid{G}\;\Varid{mu})\comp i_2\mathrel{=}\mathbf{in}\comp i_2\comp \Conid{G}\;\Varid{mu}}
\just\equiv{ natural-i2, cancelamento-+ }
\ensuremath{\mathbf{in}\comp i_2\comp \Conid{G}\;\Varid{mu}\mathrel{=}\mathbf{in}\comp i_2\comp \Conid{G}\;\Varid{mu}}
\just\equiv{ P.R.I }
\ensuremath{\Conid{True}}
\qed
\end{eqnarray*}

\emph{Maps}:
\emph{Monad}:
\emph{Let expressions}:

A função \ensuremath{\Varid{let\char95 exp}} é responsável por substituir todas as variáveis numa expressão \ensuremath{\Conid{Expr}} pelas suas correspondentes expressões
atribuídas por uma função fornecida como argumento.

A definição da função \ensuremath{\Varid{let\char95 exp}} utiliza o catamorfismo \ensuremath{\Varid{cataExpr}}. No caso de encontrar uma variável (\ensuremath{\Conid{V}}), 
a função \ensuremath{\Varid{f}} é usada para substituir essa variável pela expressão correspondente. Para valores 
numéricos (\ensuremath{\Conid{N}}), a função mantém o valor inalterado. Para operadores (caso \ensuremath{\Conid{T}}), a função constrói uma nova expressão 
que combina os resultados das subexpressões recursivamente.

Em suma, \ensuremath{\Varid{let\char95 exp}} avalia uma expressão ao substituir todas as variáveis pelas expressões correspondentes, garantindo 
que a estrutura da expressão original é mantida. De seguida, o diagrama mostra como a operação do catamorfismo percorre e 
transforma a expressão.

\begin{eqnarray*}
\xymatrix@C=2cm{
    \ensuremath{\Conid{Expr}\;\Conid{C}\;\Conid{A}}
           \ar[d]_-{\ensuremath{\Varid{let\char95 exp}\;\Varid{f}}}
           \ar@/_/[r]_-{\ensuremath{\Varid{out}}_\ensuremath{\Conid{Expr}}}
&
    \ensuremath{\Conid{A}\mathbin{+}(\Conid{C}\mathbin{+}(\Conid{Op}\times(\Conid{Expr}\;\Conid{C}\;\Conid{A})}^*))
           \ar[d]^{\ensuremath{\Varid{id}\mathbin{+}(\Varid{id}\mathbin{+}(\Varid{id}\times\map \;(\Varid{let\char95 exp}\;\Varid{f})))}}
           \ar@/_/[l]_-{\ensuremath{\mathbf{in}}_\ensuremath{\Conid{Expr}}}
\\
    \ensuremath{\Conid{Expr}\;\Conid{C}\;\Conid{B}}
&
    \ensuremath{\Conid{A}\mathbin{+}(\Conid{C}\mathbin{+}(\Conid{Op}\times(\Conid{Expr}\;\Conid{C}\;\Conid{B})}^*))
           \ar[l]^-{\ensuremath{\alt{\Varid{f}}{\alt{\Conid{N}}{\uncurry{\Conid{T}}}}}}
}
\end{eqnarray*}

Segue a implementação da função \ensuremath{\Varid{let\char95 exp}}:

\begin{hscode}\SaveRestoreHook
\column{B}{@{}>{\hspre}l<{\hspost}@{}}%
\column{E}{@{}>{\hspre}l<{\hspost}@{}}%
\>[B]{}\Varid{let\char95 exp}\;\Varid{f}\mathrel{=}\Varid{cataExpr}\;\alt{\Varid{f}}{\alt{\Conid{N}}{\uncurry{\Conid{T}}}}{}\<[E]%
\ColumnHook
\end{hscode}\resethooks

Catamorfismo monádico:
\begin{hscode}\SaveRestoreHook
\column{B}{@{}>{\hspre}l<{\hspost}@{}}%
\column{5}{@{}>{\hspre}l<{\hspost}@{}}%
\column{E}{@{}>{\hspre}l<{\hspost}@{}}%
\>[B]{}\Varid{mcataExpr}\;\Varid{g}\mathrel{=}\Varid{g}\mathbin{.!}(\Varid{dl'}\comp \Varid{recExpr}\;(\Varid{mcataExpr}\;\Varid{g})\comp \Varid{outExpr}){}\<[E]%
\\[\blanklineskip]%
\>[B]{}\Varid{dl'}\mathbin{::}\Conid{Monad}\;\Varid{m}\Rightarrow \Varid{a}+(\Varid{b}+(\Conid{Op},[\mskip1.5mu \Varid{m}\;\Varid{c}\mskip1.5mu]))\to \Varid{m}\;(\Varid{a}+(\Varid{b}+(\Conid{Op},\Varid{m}\;[\mskip1.5mu \Varid{c}\mskip1.5mu]))){}\<[E]%
\\
\>[B]{}\Varid{dl'}\mathrel{=}\alt{\Varid{return}\comp i_1}{\alt{\Varid{return}\comp i_2\comp i_1}{\Varid{aux}}}{}\<[E]%
\\
\>[B]{}\hsindent{5}{}\<[5]%
\>[5]{}\mathbf{where}\;\Varid{aux}\;(\Varid{op},\Varid{ms})\mathrel{=}\mathbf{do}\;\Varid{m}\leftarrow \Varid{lamb}\;\Varid{ms};(\Varid{return}\comp i_2\comp i_2)\;(\Varid{op},\Varid{return}\;\Varid{m}){}\<[E]%
\ColumnHook
\end{hscode}\resethooks


Avaliação de expressões:

A função \ensuremath{\Varid{evaluate}} avalia expressões do tipo \ensuremath{\Conid{Expr}} e retorna o resultado da avaliação com o tipo \ensuremath{\Conid{Maybe}}.
Esta função tem em conta dois cenários de erro: variáveis não instanciadas e operadores aplicados a um número incorreto de argumentos.

A função \ensuremath{\Varid{evaluate}} utiliza o catamorfismo \ensuremath{\Varid{mcataExpr}} para avaliar a expressão. Este conceito generaliza o conceito de catamorfismo simples para permitir trabalhar em contextos monádicos.
O ponto central deste conceito é que combina a lógica de transformação (\ensuremath{\Varid{g}}) com a recursão da estrutura, enquanto lida automaticamente com contextos monádicos. Assim, evitámos
tratar manualmente de cada contexto monádico \ensuremath{\Conid{Maybe}} em cada passo.

No caso do \ensuremath{\Varid{evaluate}}, o gene \ensuremath{\Varid{gene}} define como é que se processa cada nodo da estrutura \ensuremath{\Conid{Expr}}.
O gene \ensuremath{\Varid{gene}} lida com três casos principais:
\begin{itemize}
\item \ensuremath{\Conid{V}} : Para uma variável, retornámos \ensuremath{\Conid{Nothing}}, porque as variávis não podem ser avaliadas.
\item \ensuremath{\Conid{N}} : Para um número, retornámos o próprio número com \ensuremath{\Conid{Just}}.
\item \ensuremath{\Conid{T}} : Para um operador, utilizámos a função auxiliar \ensuremath{\Varid{aux}} que:
    \begin{itemize}
        \item avalia todos os argumentos no contexto \ensuremath{\Conid{Maybe}}, isto é, verifica se todos os argumentos são válidos;
        \item aplica a função \ensuremath{\Varid{result}} para calcular o resultado final, caso todos os argumentos sejam válidos.
    \end{itemize}
\end{itemize}

A função \ensuremath{\Varid{result}} define o comportamento esperado para cada operados e valida a aridade, assim garante que apenas os operadores com a aridade correta sejam avaliados.
Caso contrário, a avaliação falha e retorna \ensuremath{\Conid{Nothing}}.

Segue a implementação da função \ensuremath{\Varid{evaluate}}:

\begin{hscode}\SaveRestoreHook
\column{B}{@{}>{\hspre}l<{\hspost}@{}}%
\column{5}{@{}>{\hspre}l<{\hspost}@{}}%
\column{E}{@{}>{\hspre}l<{\hspost}@{}}%
\>[B]{}\Varid{evaluate}\mathrel{=}\Varid{mcataExpr}\;\Varid{gene}{}\<[E]%
\\[\blanklineskip]%
\>[B]{}\Varid{gene}\mathbin{::}(\Conid{Num}\;\Varid{a},\Conid{Ord}\;\Varid{a})\Rightarrow \Varid{b}+(\Varid{a}+(\Conid{Op},\Conid{Maybe}\;[\mskip1.5mu \Varid{a}\mskip1.5mu]))\to \Conid{Maybe}\;\Varid{a}{}\<[E]%
\\
\>[B]{}\Varid{gene}\mathrel{=}\alt{\underline{\Conid{Nothing}}}{\alt{\Conid{Just}}{\Varid{aux}}}{}\<[E]%
\\
\>[B]{}\hsindent{5}{}\<[5]%
\>[5]{}\mathbf{where}\;\Varid{aux}\;(\Varid{op},\Varid{args})\mathrel{=}\mathbf{do}\;\Varid{argsR}\leftarrow \Varid{args};\Varid{result}\;\Varid{op}\;\Varid{argsR}{}\<[E]%
\\[\blanklineskip]%
\>[B]{}\Varid{result}\mathbin{::}(\Conid{Num}\;\Varid{a},\Conid{Ord}\;\Varid{a})\Rightarrow \Conid{Op}\to [\mskip1.5mu \Varid{a}\mskip1.5mu]\to \Conid{Maybe}\;\Varid{a}{}\<[E]%
\\
\>[B]{}\Varid{result}\;\Conid{Add}\;[\mskip1.5mu \Varid{x},\Varid{y}\mskip1.5mu]\mathrel{=}\Conid{Just}\;(\Varid{x}\mathbin{+}\Varid{y}){}\<[E]%
\\
\>[B]{}\Varid{result}\;\Conid{Mul}\;[\mskip1.5mu \Varid{x},\Varid{y}\mskip1.5mu]\mathrel{=}\Conid{Just}\;(\Varid{x}\mathbin{*}\Varid{y}){}\<[E]%
\\
\>[B]{}\Varid{result}\;\Conid{Suc}\;[\mskip1.5mu \Varid{x}\mskip1.5mu]\mathrel{=}\Conid{Just}\;(\Varid{x}\mathbin{+}\mathrm{1}){}\<[E]%
\\
\>[B]{}\Varid{result}\;\Conid{ITE}\;[\mskip1.5mu \Varid{cond},\Varid{t},\Varid{f}\mskip1.5mu]\mathrel{=}\mathbf{if}\;\Varid{cond}\not\equiv \mathrm{0}\;\mathbf{then}\;\Conid{Just}\;\Varid{t}\;\mathbf{else}\;\Conid{Just}\;\Varid{f}{}\<[E]%
\\
\>[B]{}\Varid{result}\;\anonymous \;\anonymous \mathrel{=}\Conid{Nothing}{}\<[E]%
\ColumnHook
\end{hscode}\resethooks

%----------------- Índice remissivo (exige makeindex) -------------------------%

\printindex

%----------------- Bibliografia (exige bibtex) --------------------------------%

\bibliographystyle{plain}
\bibliography{cp2425t}

%----------------- Fim do documento -------------------------------------------%
\end{document}
